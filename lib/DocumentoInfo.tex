%%%%%%%%%%%%%%%%%%%%%%%%%%%%%%%%%%%%%%%%%%%%%%%%%%
%%                                              %%
%%        Comandos usados no documento          %%
%%                                              %%
%%%%%%%%%%%%%%%%%%%%%%%%%%%%%%%%%%%%%%%%%%%%%%%%%%
% Capa e Folha de Rosto e Folha de Aprovação
\renewcommand{\imprimircapa}{
%%%%%%%%%%%%%%%%%%%%%%%%%%%%%%%%%%%%%%%%%%%%%%%%%%
%%                                              %%
%%              Início da Capa                  %%
%%                                              %%
%%%%%%%%%%%%%%%%%%%%%%%%%%%%%%%%%%%%%%%%%%%%%%%%%%

\thispagestyle{empty}

\begin{figure}
    \centering
    \includegraphics[scale=0.2]{./lib/logoifrncn.jpg}
\end{figure}

\begin{center}
    \large{\textbf{\imprimirinstituicao}}

    \vspace{\stretch{2}}
    
    \large{\MakeUppercase{\textbf{\imprimirautor}}}
    
    \vspace{\stretch{2}}
    
    \large{\MakeUppercase{\textbf{\imprimirtitulo \imprimirsubtitulo}}}
    
    \vspace{\stretch{3}}
    
    \large{\textbf{\imprimirlocal}} \\
    \large{\textbf{\imprimirdata}}
\end{center}

\newpage
%%%%%%%%%%%%%%%%%%%%%%%%%%%%%%%%%%%%%%%%%%%%%%%%%%
%%                                              %%
%%                  Fim da Capa                 %%
%%                                              %%
%%%%%%%%%%%%%%%%%%%%%%%%%%%%%%%%%%%%%%%%%%%%%%%%%%
}
\renewcommand{\imprimirfolhaderosto}{\input{pre-textuais/FolhaDeRosto}}

% subtítulo
\providecommand{\imprimirsubtitulo}{}
\newcommand{\subtitulo}[1]{\renewcommand{\imprimirsubtitulo}{#1}}

% criar fonte em imagens, tabelas e quadros
\newcommand{\source}[1]{\legend{\textbf{Fonte:} {#1}}}    

% texto em arial
%\renewcommand{\sfdefault}{phv}
%\renewcommand{\rmdefault}{phv}

% texto em Times
%\renewcommand{\sfdefault}{ptm}
%\renewcommand{\rmdefault}{ptm}

%%%%%%%%%%%%%%%%%%%%%%%%%%%%%%%%%%%%%%%%%%%%%%%%
%%          Informações do documento          %%
%%%%%%%%%%%%%%%%%%%%%%%%%%%%%%%%%%%%%%%%%%%%%%%%


% Preencha com os seus dados
\titulo{Relatório Sobre Independência Básica de Sistema Experimental para Com Suas Dependências: Abordagem prática sobre o PHP}
\tipotrabalho{Monografia (bacharel em Sistema de Informação) }
\autor{LUIZ PAULO DE LIMA ARAÚJO}
\orientador{Merciosvaldo da Silva Exemplo}     % nível acadêmico + nome
\local{CURRAIS NOVOS - RN}
\data{\the\year}
\instituicao{
    INSTITUTO FEDERAL DE EDUCAÇÃO, CIÊNCIA E TECNOLOGIA\par
    DO RIO GRANDE DO NORTE\par
    CAMPUS CURRAIS NOVOS\par
}

% Rótulo do orientador
\renewcommand{\imprimirorientadorRotulo}{Orientador(a): Merciosvaldo da Silva Exemplo }
\renewcommand{\imprimircoorientadorRotulo}{Co-orientador(a): }

\preambulo{Trabalho de conclusão de curso apresentado ao curso de graduação em Tecnologia em Sistemas para Internet, como parte dos requisitos para obtenção do título de Tecnólogo em Sistemas para Internet pelo Instituto Federal do Rio Grande do Norte.}

%%%%%%%%%%%%%%%%%%%%%%%%%%%%%%%%%%%%%%%%%%%%%%%%%%
%%                                              %%
%%        Comandos usados no documento          %%
%%                                              %%
%%%%%%%%%%%%%%%%%%%%%%%%%%%%%%%%%%%%%%%%%%%%%%%%%%

%%%%%%%%%%%%%%%%%%%%%%%%%%%%%%%%%%
%%                              %%
%%            Tabelas           %%
%%                              %%
%%%%%%%%%%%%%%%%%%%%%%%%%%%%%%%%%%
\newcommand{\quadroname}{Quadro}

\newfloat[chapter]{quadro}{loq}{\quadroname}
\newlistof{listofquadros}{loq}{\listofquadrosname}
\newlistentry{quadro}{loq}{0}

% configurações para atender às regras da ABNT
\setfloatadjustment{quadro}{\centering}
\counterwithout{quadro}{chapter}
\renewcommand{\cftquadroname}{\quadroname\space} 
\renewcommand*{\cftquadroaftersnum}{\hfill--\hfill}

\setfloatlocations{quadro}{hbtp}


%%%%%%%%%%%%%%%%%%%%%%%%%%%%%%%%%%%%%%%%%%%%%%%%%%
%%                                              %%
%%   Configurações de aparência do PDF final    %%
%%                                              %%
%%%%%%%%%%%%%%%%%%%%%%%%%%%%%%%%%%%%%%%%%%%%%%%%%%

%%%%%%%%%%%%%%%%%%%%%%%%%%%%%%%%%%
%%                              %%
%%      Informações do PDF      %%
%%                              %%
%%%%%%%%%%%%%%%%%%%%%%%%%%%%%%%%%%
\makeatletter
\hypersetup{
    %  	pagebackref=true,
		pdftitle={\@title}, 
		pdfauthor={\@author},
        pdfsubject={\imprimirpreambulo},
        pdfcreator={LaTeX with abnTeX2},
		pdfkeywords={abnt}{latex}{abntex}{abntex2}{trabalho acadêmico}, 
    %   false: boxed links; true: colored links
		colorlinks=true,
    % 	color of internal links
        linkcolor=black,
    %   color of links to bibliography
        citecolor=black,
    %   color of file links
        filecolor=magenta,
		urlcolor=blue,
		bookmarksdepth=4
}
\makeatother

%%%%%%%%%%%%%%%%%%%%%%%%%%%%%%%%%%
%%                              %%
%%         Indentação           %%
%%                              %%
%%%%%%%%%%%%%%%%%%%%%%%%%%%%%%%%%%

% O tamanho do parágrafo é dado por:
\setlength{\parindent}{1.5cm}

% Controle do espaçamento entre um parágrafo e outro:
\setlength{\parskip}{0.2cm}  % tente também \onelineskip


%%%%%%%%%%%%%%%%%%%%%%%%%%%%%%%%%%
%%                              %%
%%         Algoritmos           %%
%%                              %%
%%%%%%%%%%%%%%%%%%%%%%%%%%%%%%%%%%

% Algoritmos
% \floatname{algorithm}{Algoritmo}%Algoritmo
% \renewcommand{\listalgorithmname}{LISTA DE ALGORITMOS}

\renewcommand{\algorithmicindent}{3.0em}
\renewcommand{\algorithmicrequire}{\textbf{entrada}}
\renewcommand{\algorithmicensure}{\textbf{garanta}}
\renewcommand{\algorithmicreturn}{\textbf{retorne}}
\renewcommand{\algorithmicor}{\textbf{ou}}
\renewcommand{\algorithmicand}{\textbf{e}}
\renewcommand{\algorithmicend}{\textbf{fim}}
\renewcommand{\algorithmicif}{\textbf{se}}
\renewcommand{\algorithmicelse}{\textbf{sen\~ao}}
\renewcommand{\algorithmicthen}{\textbf{ent\~ao}}
\renewcommand{\algorithmicfor}{\textbf{para}}
\renewcommand{\algorithmicforall}{\textbf{para cada}}
\renewcommand{\algorithmicrepeat}{\textbf{repita}}
\renewcommand{\algorithmicuntil}{\textbf{até que}}
\renewcommand{\algorithmicwhile}{\textbf{enquanto}}
\renewcommand{\algorithmicdo}{\textbf{fa\c{c}a}}

