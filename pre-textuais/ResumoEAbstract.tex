%%%%%%%%%%%%%%%%%%%%%%%%%%%%
%%                        %%
%%         Resumo         %%
%%                        %%
%%%%%%%%%%%%%%%%%%%%%%%%%%%%

% Definir a fonte padrão para Helvetica

% Modificar apenas o título "Resumo" para negrito
{\fontsize{12pt}{12pt}\selectfont % Início da modificação para tamanho 12pt
\renewcommand{\resumoname}{\textbf{\fontsize{12pt}{12pt}\selectfont RESUMO}} % Título do resumo em negrito e tamanho 12pt

\begin{resumo}
   \noindent 
    %O resumo tem a função de resumir os pontos-chave da monografia, 
    %apresentando sucintamente a introdução, metodologia, resultados,  
    %discussão e conclusões, além de destacar a relevância do estudo. 
    %Deve ser conciso, informativo e atrativo, com uma extensão usualmente 
    %entre 150 e 300 palavras, dependendo das diretrizes da instituição 
    %ou da revista acadêmica.

   %Os computadores emergem no século XX como máquinas que possibilitaram
   %a evolução da forma como se lida com a informação gerada nas sociedades
   %humanas modernas. O aumento vertiginoso do poder computacional permitiu
   %a consolidação de sistemas cada vez mais complexos o que implicou na
   %mudança na forma como o software é escrito. De programas inteiramente
   %concebidos por um programador para grandes sistemas mantidos por vários técnicos
   %cujo os alicerçes nascem externamente na forma de módulos e pacotes.
   %Entretanto, a agilidade propiciada pela modularidade trouxe consigo o
   %possível uso incauteloso das dependências que o fazem ser funcional
   %implicando e problemas que acomulam-se ao longo do tempo. Partindo disso
   %, criou-se uma aplicação experimental orientada a objetos voltada para
   %ateliês de costura utilizando a arqutetura hexagonal com classes adaptadoras
   %como forma de inverter dependências nas partes de comunicação e dados. Por fim,
   %concluiu-se que os esforços em prol da independência entre sistema e dependências
   %é possível como também é dispendioso em tempo e recursos intelectuais ao
   %mesmo tempo que e é capaz de causar \textit{overengeneering}.

   Os computadores emergem no século XX como máquinas que possibilitaram
   a evolução da forma como se lida com a informação gerada nas sociedades
   humanas modernas. O aumento vertiginoso do poder computacional permitiu
   a consolidação de sistemas cada vez mais complexos o que implicou na
   mudança na forma como o software é escrito. De programas inteiramente
   concebidos por um programador para grandes sistemas mantidos por vários técnicos
   cujo os alicerçes nascem externamente na forma de módulos e pacotes.
   Entretanto, a agilidade propiciada pela modularidade trouxe consigo o
   possível uso incauteloso das dependências que o fazem ser funcional
   implicando e problemas que acomulam-se ao longo do tempo. Partindo disso
   , criou-se uma aplicação experimental orientada a objetos voltada para
   ateliês de costura utilizando a arqutetura hexagonal com classes adaptadoras
   como forma de inverter dependências nas partes de comunicação e dados. Por fim,
   concluiu-se que os esforços em prol da independência entre sistema e dependências
   é possível como também é dispendioso em tempo e recursos intelectuais ao
   mesmo tempo que e é capaz de causar o \textit{overengeneering}.

    \textbf{Palavras Chave: Orientação a Objetos, Gestão de Dependência, Padrões, Arquitetura}: 
\end{resumo}
}
\newpage

%%%%%%%%%%%%%%%%%%%%%%%%%%%%
%%                        %%
%%        Abstract        %%
%%                        %%
%%%%%%%%%%%%%%%%%%%%%%%%%%%%

% Modificar apenas o título "Resumo" para negrito
{\fontsize{12pt}{12pt}\selectfont % Início da modificação para tamanho 12pt
\renewcommand{\abstractname}{\textbf{\fontsize{12pt}{12pt}\selectfont ABSTRACT}} % Título do resumo em negrito e tamanho 12pt

\begin{abstract}
    \noindent
  %The abstract serves the purpose of summarizing the key points of the thesis,
  %briefly presenting the introduction, methodology, results, discussion, and
  %conclusions, while highlighting the study's relevance. It should be concise,
  %informative, and engaging, typically ranging from 150 to 300 words, depending
  %on the guidelines of the institution or academic journal.
    
    \textbf{Keywords}:
\end{abstract}
}
\newpage
