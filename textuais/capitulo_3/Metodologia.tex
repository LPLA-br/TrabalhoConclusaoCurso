
%%%%%%%%%%%%%%%%%%%%%%%%%%%
%%      Metodologia      %%
%%%%%%%%%%%%%%%%%%%%%%%%%%%

\section{\textbf{METODOLOGIA}}
    \label{cha:metodologia}

    %A metodologia em um trabalho acadêmico é a seção que explica
    %como a pesquisa foi conduzida, incluindo os métodos, técnicas
    %e ferramentas usadas para coletar dados e analisar informações.
    %Essa parte detalha o caminho seguido para atingir os objetivos
    %do estudo, incluindo abordagens qualitativas, quantitativas ou
    %mistas, os procedimentos de coleta de dados e análise, além
    %dos critérios de seleção da amostra. É crucial para a
    %validade e confiabilidade dos resultados.
    
    % propriedades: ferramentas, métodos e técnicas

    % introdução da metodologia
    O presente segmento tem o objetivo de apresentar quais métodos presentes em
    obras técnicas serão selecionados para possibilitar a consolidação de um
    processo funcional na construção do objeto experimental deste trabalho.

    Assim, ao se posicionar como objetivo nuclear a independência estrita do
    software para com suas dependências, em observância do que se foi definido
    como meio para atingir tal estado nas obras de grandes autores, conforme
    citado anteriormente no referencial teórico, parte-se então para uma
    proposta metodológica simples, porém efetiva.

    %etapas
    A principio todo \textit{software} surgirá, de acordo com
    \citeonline[p.244]{rogerPressman2021}, em conformidade com o processo de
    levantamento dos requisitos funcionais. Dessa maneira, ele aponta estas
    etapas como sendo marco inicial de todo esforço de engenharia de sistemas,
    que decorre sobre uma comunicação constante de partes interessadas.

    %partes envolvidas
    Indo além, caracteriza-se como partes interessadas: clientes; funcionários;
    gerentes; administradores e, nos casos em que o sistema já é usado no
    estado em que se encontra, usuários. Todos eles contribuintes de um
    processo que resultará em situações ideais, num sistema possuidor de
    total utilidade ao empreendimento para o qual fora desenvolvido.

    Desta forma, partindo da primeira etapa metodológica comunicativa,
    \citeonline{rogerPressman2021, somervilleIam2011} pontuam mais outras quatro
    atividades fundamentais a qualquer projeto ativamente desenvolvido, onde
    lista:

    \begin{itemize}
      %\item{ Comunicação \- Partes comunicam-se para conhecimento do que deve ser implementado. }
      \item{ Planejamento - Organização de tempo e recursos; }
      \item{ Modelagem -  Arquitetação e modelagem da estrutura do sistema; }
      \item{ Construção - Escrita/Implementação do sistema; }
      \item{ Entrega - Testes, implantação, monitoramento, administração e preparação para próxima iteração; }
    \end{itemize}

    %fluxo das etapas/tempo
    Neste sentido, partindo de um conjunto bem definido e aparentemente linear
    de tarefas, encontramos que elas não são sequencialmente ótimas, pois
    apresentam uma organização temporal muito simplista. Logo, tendo em vista
    que todo esforço técnico pode se desenrolar em finitas organizações
    temporalmente válidas, os fluxos de processos apontados pelo mesmo autor
    dinamizar-se-ão como demonstrado na próxima figura.

    \begin{figure}[h]
      \centering
      \caption{ Fluxos de Processos. }
      \includegraphics[scale=0.5]{lib/atividades-processo.png}
      \source{ Pressman 2021 {\imprimirdata}}
      \label{fig:figura}
    \end{figure}

    Diante dessa premissa, admitiu-se uma flexibilização na execução de etapas
    mais próxima de uma iteração entre elas e ou de todas elas (evolutivo).
    Em relação aos prazos formais não foram admitidos pelo fato de que a aplicação
    provavelmente não será utilizada de fato.

    % para planejamento e modelagem (subconjunto duas dentre todas etapas)
    Consequentemente além das etapas iniciais de comunicação e planejamento,
    dentro do modelo de etapas e fluxo flexível ora apresentado, segue-se
    iteradamente para modelagem e construção que determinam, sob a orientação
    de arquiteturas e padrões comuns, características significativas em impacto
    por todo tempo de vida útil do \textit{software}.

    No que diz respeito à arquitetura, assume-se, de modo a evitar a
    complexidade de abordagens menos intuitivas, o modelo proposto por
    \citeonline{alistairHexagonal2005} que aponta o uso do padrão de projeto
    \textit{Adapter} como forma de isolar os objetos do domínio de negócios dos
    demais objetos do ambiente referido com \textit{Outside World} "Mundo
    Externo". A tal arquitetura atribui-se o nome de \textit{Hexagonal Architecture}.

    \begin{figure}[h]
      \centering
      \caption{ Interação entre Classes Nucleares do Domínio o \textit{Outside World}. }
      \includegraphics[scale=0.4]{lib/deployment-uml.png}
      \source{ O Autor {\imprimirdata}}
      \label{fig:figura}
    \end{figure}

    No entanto, exclui-se, de modo a seguir a linha do experimental, a parte
    desenvolvida que provê interface ao usuário final. Logo resta apenas a
    escrita do domínio de negócios, do adaptador de banco de dados e do
    adaptador para comunicação, onde usar-se-á protocolo da camada de
    transporte em redes de computadores TCP/IP.

    No tocante, ao campo das classes em sistemas \ac{oop}, há interrelações entre
    elas na forma de herança ou composição. Entretanto, ocorre que, em meios
    técnicos, discute-se acerca da necessidade de se priorizar a primeira sobre
    a segunda \cite{rodriguesLenon2023}. Não obstante, o objeto deste trabalho
    assume uma abordagem mista onde uma e outra são empregadas simultaneamente
    de acordo com o que se julga mais adequado e conveniente ao contexto de
    implementação.

    Quanto a objetos que agregam ou são compostos de N objetos, adotou-se
    o padrão inicialmente definido por \citeonline{goldberg1984smalltalk}.
    Padrão este que consiste em separar responsabilidade de gerenciar
    uma coleção de sub objetos de seu objeto manipulador principal assim
    evitando violações do princípio da responsabilidade única.

    Em suma, de modo a prosseguir mais proximamente ao domínio de negócios,
    caberá mais apropriadamente ao desenvolvimento, tratar acerca dos detalhes
    do domínio que orientou a implementação do experimento correspondente
    aos objetivos.
    
    %\input{textuais/capitulo_3/sub-topicos/QuestoesDePesquisa}
    %\input{textuais/capitulo_3/sub-topicos/PropostaMetodologica}
    %\input{textuais/capitulo_3/sub-topicos/Atividades}

\newpage
