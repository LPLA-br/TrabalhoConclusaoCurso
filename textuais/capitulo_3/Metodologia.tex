
%%%%%%%%%%%%%%%%%%%%%%%%%%%
%%      Metodologia      %%
%%%%%%%%%%%%%%%%%%%%%%%%%%%

\section{\textbf{METODOLOGIA}}
    \label{cha:metodologia}

    %A metodologia em um trabalho acadêmico é a seção que explica
    %como a pesquisa foi conduzida, incluindo os métodos, técnicas
    %e ferramentas usadas para coletar dados e analisar informações.
    %Essa parte detalha o caminho seguido para atingir os objetivos
    %do estudo, incluindo abordagens qualitativas, quantitativas ou
    %mistas, os procedimentos de coleta de dados e análise, além
    %dos critérios de seleção da amostra. É crucial para a
    %validade e confiabilidade dos resultados.
    
    % propriedades: ferramentas, métodos e técnicas
    % abordagem mista.

    Todo software surgirá, de acordo com \citeonline[p. 244]{rogerPressman2021}
    , em conformidade com o processo fundamental de levantamento de requisitos.
    Ele define esse processo como sendo o precessor de qualquer esforço
    prático de engenharia.

    À esta etapa contribui plúrimas pessoas como clientes, gerentes e usuários
    de um ambiente de negócios. Objetiva-se estabelecer os requisitos nucleares
    uma ferramenta capaz de resolver um ou vários problemas informacionais
    crônicos do negócio.

    E foi a partir da etapa fundamental de levantamento de requisitos que se
    inicia as etapas de desenvolvimento de uma aplicação voltada para micro
    empreendedoras da área de confecção têxtil.

    Partiu-se do pressuposto de que todo empreendimento de costura informal
    poderiar possuir problemas comuns de ingerência de suas informações. A
    partir deste pressuposto estabeleceu-se um canal de comunicação coloquial
    com profissional com anos de experiência de atuação autônoma próxima
    ao autor.

    A comunicação inicial revelou situações problemáticas onde: os prazos são
    esquecidos, o levantamento de custos com insumos usados em confecções e
    concertos é ignorado e armazenamento irregular de medidas corporais por
    vezes dificulta consultas rápidas e confunde quando em consultado
    dados obsoletos.

    Quanto aos requisitos não funcionais, concluiu-se que sistema atuante em
    redes locais seria, em materia de confiabilidade, custo e legalidade, mais
    adequado para estes ambientes de processos simples porém fundamentais.

    \begin{figure}[h]
      \centering
      \caption{ Diagrama da Infraestrutura Local de Rede. }
      \includegraphics[scale=0.5]{lib/nodes-uml.png}
      \source{ O Autor {\imprimirdata}}
      \label{fig:figura}
    \end{figure}

    Seguindo a etapa metodológica comunicativa, \citeonline{rogerPressman2021}
    define outras quatro atividades metodologicas fundamentais a todo projeto
    ativo de software:

    \begin{itemize}
      \item{ Comunicação \- Partes comunicam-se para conhecimento do que deve ser implementado. }
      \item{ Planejamento \- Parte técnica organiza-se em questão de tempo e recursos. }
      \item{ Modelagem \- Parte técnica modela o sistema. }
      \item{ Construção \- Parte técnica implementa o sistema. }
      \item{ Entrega \- Testes, implantação e monitoramento mais preparação para proxima iteração. }
    \end{itemize}

    E também o fluxo como tais atividades fundamentais
    podem ser executadas temporalmente:

    \begin{figure}[h]
      \centering
      \caption{ Fluxos de Processos Adotados. }
      \includegraphics[scale=0.5]{lib/atividades-processo.png}
      \source{ Pressman 2021 {\imprimirdata}}
      \label{fig:figura}
    \end{figure}

    A solução desenvolvida configurou-se, em questão de ordem dos processos,
    como sendo semi linear com capacidade para iteração entre etapas internas
    ou iteração de todas etapas. Entretanto não foi planejado prazos para
    construção dos componentes do software.

    Arquiteturalmente a aplicação adota um modelo em camadas onde cada camada é
    responsável por uma macro-responsabilidade definida ainda no levantamento
    de requisitos.

    A garantia de separação entre domínio e funcionalidades satélites dar-se
    por orientação da \textit{Hexagonal Architecture} que faz uso massivo e
    simples de classes adaptadoras que fornecem uma ponte entre o
    \textit{software} desenvolvido e seu meio necessário à utilidade.

    % CLASSIFICAÇÃO EVANS DDD
    % Entidades, ValueObjects, Services, Packages.
    % Aggregates, Factories, Repositories.
    O modelo acamadado adotado orienta-se, com relação a classificação
    de objetos, sobre o trabalho de \citeonline{evans2004domain}

    \begin{figure}[h]
      \centering
      \caption{ Diagrama Deployment Para Componentes Possuidores de Várias Classes. }
      \includegraphics[scale=0.4]{lib/deployment-uml.png}
      \source{ O Autor {\imprimirdata}}
      \label{fig:figura}
    \end{figure}





    % DESMARQUE SE QUISER READICIONÁ-LOS

    %\input{textuais/capitulo_3/sub-topicos/QuestoesDePesquisa}
    
    %\input{textuais/capitulo_3/sub-topicos/PropostaMetodologica}
    
    %\input{textuais/capitulo_3/sub-topicos/Atividades}

\newpage
