
%%%%%%%%%%%%%%%%%%%%%%%%%%%
%%      Metodologia      %%
%%%%%%%%%%%%%%%%%%%%%%%%%%%

\section{\textbf{METODOLOGIA}}
    \label{cha:metodologia}

    %A metodologia em um trabalho acadêmico é a seção que explica
    %como a pesquisa foi conduzida, incluindo os métodos, técnicas
    %e ferramentas usadas para coletar dados e analisar informações.
    %Essa parte detalha o caminho seguido para atingir os objetivos
    %do estudo, incluindo abordagens qualitativas, quantitativas ou
    %mistas, os procedimentos de coleta de dados e análise, além
    %dos critérios de seleção da amostra. É crucial para a
    %validade e confiabilidade dos resultados.
    
    % propriedades: ferramentas, métodos e técnicas
    % abordagem mista.

    O presente trabalho adota o procedimentoo experimental sob método dedutivo
    onde, a partir de um experimento real, deduzir-se-á uma conclusão
    sobre o atual estado da arte em técnicas de escrita de software
    sobre o paradigma orientado a objetos. Tudo isto respeitando
    limites e viezes impostos pela a ótica e experiência do autor.

    Os procedimentos metodológicos técnicos adotados pela parte prática
    da pesquisa são providos por obras pertinentes ao campo da
    engenharia de software. Princípios, padrões e fundamentos são
    aplicados para reforço e avalição de práticas.

    O projeto prático consiste de uma aplicação para redes TCP/IP. Trata-se de
    um software cujas funcionalidades atendem necessidades de controle e
    monitoramento de recursos e e processos em ambiente de negócios de pequeno
    impacto no ramo da produção de peças de vestuário.

    \begin{figure}[h]
      \centering
      \caption{ Diagrama Deployment Infraestrutura Local de Rede. }
      \includegraphics{lib/nodes-uml.png}
      \source{ O Autor {\imprimirdata}}
      \label{fig:figura}
    \end{figure}

    A obra \citeonline[p. ~83]{rogerPressman2021} define o software como sendo
    um processo iterativo de aprendizado e resultado. Assim como a pesquisa
    científica, o processo possui atividades metodológicas com características
    geralmente iguais.

    \begin{itemize}
      \item{ Comunicação \- Partes comunicam-se para conhecimento do que deve ser implementado. }
      \item{ Planejamento \- Parte técnica organiza-se em questão de tempo e recursos. }
      \item{ Modelagem \- Parte técnica modela o sistema. }
      \item{ Construção \- Parte técnica implementa o sistema. }
      \item{ Entrega \- Testes, implantação e monitoramento mais preparação para proxima iteração. }
    \end{itemize}

    A mesma obra apresenta fluxos fundamentais para as atividades metodológicas
    acima expostas. Dentre as quais o fluxo processual linear com possibilidade
    de iteração de etapa é adotado neste corrente trabalho.

    \begin{figure}[h]
      \centering
      \caption{ Fluxos de Processos Adotados. }
      \includegraphics{lib/atividades-processo.png}
      \source{ Pressman 2021 {\imprimirdata}}
      \label{fig:figura}
    \end{figure}

    Arquiteturalmente a aplicação adota um modelo em camadas onde cada camada é
    responsável por uma macro-responsabilidade. A garantia de separação entre
    domínio e funcionalidades satélites dar-se por orientação da \textit{Hexagonal Architecture}
    que faz uso massivo e simples de classes adaptadoras.

    % CLASSIFICAÇÃO EVANS DDD
    % Entidades, ValueObjects, Services, Packages.
    % Aggregates, Factories, Repositories.
    O modelo acamadado adotado orienta-se, com relação a classificação
    de objetos, sobre o trabalho de \citeonline{evans2004domain}

    \begin{figure}[h]
      \centering
      \caption{ Diagrama Deployment Para Componentes Possuidores de Várias Classes. }
      \includegraphics{lib/deployment-uml.png}
      \source{ O Autor {\imprimirdata}}
      \label{fig:figura}
    \end{figure}





    % DESMARQUE SE QUISER READICIONÁ-LOS

    %\input{textuais/capitulo_3/sub-topicos/QuestoesDePesquisa}
    
    %\input{textuais/capitulo_3/sub-topicos/PropostaMetodologica}
    
    %\input{textuais/capitulo_3/sub-topicos/Atividades}

\newpage
