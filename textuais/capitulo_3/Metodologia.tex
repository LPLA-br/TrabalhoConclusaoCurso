
%%%%%%%%%%%%%%%%%%%%%%%%%%%
%%      Metodologia      %%
%%%%%%%%%%%%%%%%%%%%%%%%%%%

\section{\textbf{METODOLOGIA}}
    \label{cha:metodologia}

    %A metodologia em um trabalho acadêmico é a seção que explica
    %como a pesquisa foi conduzida, incluindo os métodos, técnicas
    %e ferramentas usadas para coletar dados e analisar informações.
    %Essa parte detalha o caminho seguido para atingir os objetivos
    %do estudo, incluindo abordagens qualitativas, quantitativas ou
    %mistas, os procedimentos de coleta de dados e análise, além
    %dos critérios de seleção da amostra. É crucial para a
    %validade e confiabilidade dos resultados.
    
    % propriedades: ferramentas, métodos e técnicas

    % introdução da metodologia
    A presente seção tem o objetivo de apresentar quais métodos presentes em
    obras técnicas serão selecionados para possibilitar a consolidação de um
    processo funcional na construção do objeto experimental deste trabalho.

    Assim, ao se posicionar como objetivo nuclear a independência estrita do
    software para com suas dependências, em observância do que se foi definido
    como meio para atingir tal estado nas obras de grandes autores, conforme
    citado anteriormente no referencial teórico, parte-se então para uma
    proposta metodológica simples, porem efetiva.

    %etapas
    A principio todo \textit{software} surgirá, de acordo com
    \citeonline[p.244]{rogerPressman2021}, em conformidade com o processo de
    levantamento dos requisitos funcionais. Dessa maneira, ele aponta estas
    etapas como sendo marco inicial de todo esforço de engenharia de sistemas,
    que decorre-se sobre uma comunicação constante de partes interessadas.

    %partes envolvidas
    Indo além, caracteriza-se como partes interessadas: clientes; funcionários;
    gerentes; administradores e, nos casos em que o sistema já é usado no
    estado em que se encontra, usuários. Todos eles contribuintes de um
    processo que resultará, em situações ideais, num sistema possuidor de
    total utilidade ao empreendimento para o qual fora desenvolvido.

    Dessa forma, partindo desta primeira etapa metodológica comunicativa,
    \citeonline{rogerPressman2021} pontua mais outras quatro atividades
    fundamentais a qualquer projeto ativamente desenvolvido, onde lista:

    \begin{itemize}
      %\item{ Comunicação \- Partes comunicam-se para conhecimento do que deve ser implementado. }
      \item{ Planejamento - Organização de tempo e recursos; }
      \item{ Modelagem -  Arquitetação e modelagem da estrutura do sistema; }
      \item{ Construção - Escrita/Implementação do sistema; }
      \item{ Entrega - Testes, implantação, monitoramento, administração e preparação para proxima iteração; }
    \end{itemize}

    %fluxo das etapas
    Neste sentido, partindo de um conjunto bem definido e aparentemente linear de
    tarefas, encontramos que elas não são suficientes, pois apresentam uma organização
    temporal própria, onde também devem ser admitidas, tendo em vista que todo esforço
    técnico pode se desenrolar em finitas organizações temporalmente válidas, consolidadas
    de modo a evitar a complexidade, os fluxos de processos apontados pelo mesmo autor.

    Neste sentido, partindo de um conjunto bem definido e aparentemente linear
    de tarefas, encontramos que elas não são suficientes, pois apresentam uma
    organização temporal muito simplista, tendo em vista que todo esforço
    técnico pode se desenrolar em finitas organizações temporalmente válidas,
    consolidadas de modo a evitar a complexidade, os fluxos de processos
    apontados pelo mesmo autor são mais adequadas.

    \begin{figure}[h]
      \centering
      \caption{ Fluxos de Processos. }
      \includegraphics[scale=0.5]{lib/atividades-processo.png}
      \source{ Pressman 2021 {\imprimirdata}}
      \label{fig:figura}
    \end{figure}

    Diante dessa premissa, admitiu-se uma flexibilização na execução de etapas mais
    próxima de uma iteração entre etapas e ou de todas etapas. Logo, prazos
    formais não foram admitidos pelo fato de que a aplicação provavelente não
    será empregada para uso concreto.

    % para planejamento e modelagem
    Essa flexibilização, dar-se-á como um processo de etapas e fluxo bem
    definido e estabelecido, seguindo dentro do modelo ora apresentado, cujo
    objetivo atenderá as etapas de planejamento e modelagem, arquitetando uma macro
    característica significativa em impacto por todo o tempo de vida útil do
    \textit{software}.

    Por conseguinte, o sistema, arquiteturalmente adota uma estruturação
    usualmente conhecida sob como \textit{Hexagonal Architecture}
    onde os objetos mais importantes da aplicação, denominados de objetos
    de domínio de negócio, encontram-se isolados dos demais objetos do ambiente
    referidos como \textit{outside world} "mundo externo" \cite{alistairHexagonal2005}.
    
    Neste sentido, a garantia de separação entre domínio e funcionalidades
    utilitárias, dar-se-á por orientação da arquitetura sobrecitada que sinaliza
    o uso de classes pertecentes ao padrão de projeto \textit{Adapter} como
    meio de isolação.

    Além dessas funções, encontramos também que o sistema atende os objetivos,
    possuidores das classes que intermediam as várias características
    indispensáveis ao funcionamento da aplicação.

    \begin{figure}[h]
      \centering
      \caption{ Interação entre Classes Nucleares do Domínio o \textit{Outside World}. }
      \includegraphics[scale=0.4]{lib/deployment-uml.png}
      \source{ O Autor {\imprimirdata}}
      \label{fig:figura}
    \end{figure}

    No entanto, exclui-se, de modo a seguir a linha do experimental, a parte
    desenvolvida que provê interface ao usuário final. Logo sobrará a escrita
    do Domínio de negócios, do adaptador de banco de dados e do adaptador para
    comunicação, onde usaremos um protocolo de transporte.

    No tocante, ao campo das classes em sistemas \ac{oop}, há interelações entre
    elas na forma de herança ou composição. Entretanto, ocorre que, em meios
    técnicos, discute-se acerca da necessidade de se priorizar a primeira sobre
    a segunda \cite{rodriguesLenon2023}. Não obstante, o objeto deste trabalho
    assume uma abordagem mista onde uma e outra são empregadas simultâneamente
    de acordo com o que se julga mais adequado e conveniente ao contexto de
    implementação.

    Em suma, de modo a prosseguir mais proximamente ao domínio de negócios
    caberá, mais apropiadamente, ao desenvolvimento, tratar acerca dos detalhes
    do domínio que orientou a implementação do experimento correspondente
    aos objetivos.
    
    %\input{textuais/capitulo_3/sub-topicos/QuestoesDePesquisa}
    %\input{textuais/capitulo_3/sub-topicos/PropostaMetodologica}
    %\input{textuais/capitulo_3/sub-topicos/Atividades}

\newpage
