
%%%%%%%%%%%%%%%%%%%%%%%%%%%
%%      Metodologia      %%
%%%%%%%%%%%%%%%%%%%%%%%%%%%

\section{\textbf{METODOLOGIA}}
    \label{cha:metodologia}

    %A metodologia em um trabalho acadêmico é a seção que explica
    %como a pesquisa foi conduzida, incluindo os métodos, técnicas
    %e ferramentas usadas para coletar dados e analisar informações.
    %Essa parte detalha o caminho seguido para atingir os objetivos
    %do estudo, incluindo abordagens qualitativas, quantitativas ou
    %mistas, os procedimentos de coleta de dados e análise, além
    %dos critérios de seleção da amostra. É crucial para a
    %validade e confiabilidade dos resultados.
    
    % propriedades: ferramentas, métodos e técnicas
    % abordagem mista.

    \textit{detalhar processo que desenvolvimento executará}

    Este trabalho, em sua essência experimental, desenvolver-se-á ao entorno de
    um projeto real, aqui definido como objeto prático. Sua construção se orienta
    pela metodologia é aqui prescrita em função dos conhecimentos úteis da engenharia
    e arquitetura de \textit{software} fundamental.

    Todo software surgirá, de acordo com \citeonline[p. 244]{rogerPressman2021},
    em conformidade com o processo fundamental de levantamento de requisitos.
    Ele define esse processo como sendo o início fundamental de todo esforço
    prático de engenharia de software.

    As partes envolvidas nesta etapa são, no geral: clientes, funcionários,
    gerentes, administradores e, nos casos em que o sistema já é usado no
    estado em que se encontra, usuários. Todos eles contribuem na consolidação
    estável de um conjunto de caracteristicas que o sistema deve possuir para
    ser informacionalmente útil ao empreendimento.

    O conjunto de requisitos feito a partir da comunicação constante guia todas
    as etapas posteriores.

    Seguindo a etapa prima metodológica comunicativa,
    \citeonline{rogerPressman2021} define mais outras quatro atividades
    fundamentais a qualquer projeto ativamente desenvolvido, são elas:

    \begin{itemize}
      %\item{ Comunicação \- Partes comunicam-se para conhecimento do que deve ser implementado. }
      \item{ Planejamento - organiza-se como usar recursos durante tempo eficientemente. }
      \item{ Modelagem \-  modela-se/arquitetur-se como o sistema final estruturar-se-á. }
      \item{ Construção \- implementa-se/escreve-se o sistema. }
      \item{ Entrega \- Testes, implantação, monitoramento, administração e preparação para proxima iteração. }
    \end{itemize}

    Essas etapas foram, de modo a não tornar complexo o trabalho, incorporadas,
    em sua forma inalterada, às atividades fundamentais do objeto prático deste trabalho.

    Além de um conjunto bem definido e aparentemente linear de tarefas, uma
    organização temporal própria também deve ser adotada. Destacou-se, para a finalidade
    deste trabalho, os seguintes ordenamentos temporais:

    \begin{figure}[h]
      \centering
      \caption{ Fluxos de Processos Adotados. }
      \includegraphics[scale=0.5]{lib/atividades-processo.png}
      \source{ Pressman 2021 {\imprimirdata}}
      \label{fig:figura}
    \end{figure}

    O objeto prático temporalmente assume, na execução de suas etapas,
    seguimento semi-linear com capacidade para iteração entre etapas internas
    ou iteração de todas etapas. Não define-se limites temporais \(prazos\) nem
    estabelecimento de necessidade entrega iterativa de segmentos usáveis a
    usuários finais. Em outras palavras o projeto desenvolve-se de forma a
    possuir total conformidade com os requisitos extraidos do ambiente de
    negócios consultado.

    Determinada as etapas e a própria forma como elas são executadas,
    procede-se para como a estrutura do sistema é implementada. Aqui reside o
    cerne das questões levantadas em problemática.

    Arquiteturalmente a aplicação adota um modelo tradicional de camadas onde
    cada camada é responsável por uma responsabilidade definida ainda no
    levantamento de requisitos.

    \textit{detalhar }

    A garantia de separação entre domínio e funcionalidades satélites dar-se
    por orientação da \textit{Hexagonal Architecture} que faz uso massivo e
    simples de classes adaptadoras que fornecem uma ponte entre o
    \textit{software} desenvolvido e seu meio necessário à utilidade.

    % CLASSIFICAÇÃO EVANS DDD
    % Entidades, ValueObjects, Services, Packages.
    % Aggregates, Factories, Repositories.
    O modelo acamadado adotado orienta-se, com relação a classificação
    de objetos, sobre o trabalho de \citeonline{evans2004domain}

    \begin{figure}[h]
      \centering
      \caption{ Diagrama Deployment Para Componentes Possuidores de Várias Classes. }
      \includegraphics[scale=0.4]{lib/deployment-uml.png}
      \source{ O Autor {\imprimirdata}}
      \label{fig:figura}
    \end{figure}





    % DESMARQUE SE QUISER READICIONÁ-LOS

    %\input{textuais/capitulo_3/sub-topicos/QuestoesDePesquisa}
    
    %\input{textuais/capitulo_3/sub-topicos/PropostaMetodologica}
    
    %\input{textuais/capitulo_3/sub-topicos/Atividades}

\newpage
