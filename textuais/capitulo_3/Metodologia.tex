
%%%%%%%%%%%%%%%%%%%%%%%%%%%
%%      Metodologia      %%
%%%%%%%%%%%%%%%%%%%%%%%%%%%

\section{\textbf{METODOLOGIA}}
    \label{cha:metodologia}

    %A metodologia em um trabalho acadêmico é a seção que explica
    %como a pesquisa foi conduzida, incluindo os métodos, técnicas
    %e ferramentas usadas para coletar dados e analisar informações.
    %Essa parte detalha o caminho seguido para atingir os objetivos
    %do estudo, incluindo abordagens qualitativas, quantitativas ou
    %mistas, os procedimentos de coleta de dados e análise, além
    %dos critérios de seleção da amostra. É crucial para a
    %validade e confiabilidade dos resultados.
    
    % propriedades: ferramentas, métodos e técnicas

    % introdução da metodologia
    A presente seção tem o intuito de apresentar quais métodos, dentre os mais
    diversos presentes em obras técnicas, foram selecionados para possibilitar a
    definição de um processo funcional consolidador do objeto experimental
    deste trabalho.

    Assim, ao direcionar-se como objetivo nuclear a independência estrita do software para
    com suas dependências, em observância do que se foi definido como meio para
    atingir tal estado nas obras de grandes autores, parte-se, então, para uma
    proposta metodológica simples porem efetiva.

    %etapas
    A principio todo \textit{software} surgirá, de acordo com
    \citeonline[p.244]{rogerPressman2021}, em conformidade com o processo de
    levantamento dos requisitos funcionais. Dessa maneira, ele aponta esta
    etapa como sendo o marco inicial de todo esforço de engenharia de sistemas
    que decorre-se sobre uma comunicação constante de partes interessadas.

    %partes envolvidas
    Indo além, caracteriza-se como partes interessadas: clientes; funcionários;
    gerentes; administradores e, nos casos em que o sistema já é usado no
    estado em que se encontra, usuários. Todos eles contribuintes de um
    processo que resultará, em situações ideais, num sistema possuidor de
    total utilidade ao empreendimento para o qual fora desenvolvido.

    Dessa forma, partindo desta primeira etapa metodológica comunicativa
    \citeonline{rogerPressman2021} pontua mais outras quatro atividades
    fundamentais a qualquer projeto ativamente desenvolvido.

    \begin{itemize}
      %\item{ Comunicação \- Partes comunicam-se para conhecimento do que deve ser implementado. }
      \item{ Planejamento - Organiza-se como usar recursos durante tempo eficientemente. }
      \item{ Modelagem -  Modela-se/arquitetur-se como o sistema final estruturar-se-á. }
      \item{ Construção - Implementa-se/escreve-se o sistema. }
      \item{ Entrega - Testes, implantação, monitoramento, administração e preparação para proxima iteração. }
    \end{itemize}

    %fluxo das etapas
    Ademais, partir de um conjunto bem definido e aparentemente linear de
    tarefas não é suficiente pois uma organização temporal própria também deve
    ser admitida, tendo em visto que todo esforço técnico pode se desenrolar
    em finitas organizações temporalmente válidas. Destaca-se, de modo a evitar
    a complexidade, os fluxos de processos apontados pelo mesmo autor.

    \begin{figure}[h]
      \centering
      \caption{ Fluxos de Processos Apropriadamente Simples. }
      \includegraphics[scale=0.5]{lib/atividades-processo.png}
      \source{ Pressman 2021 {\imprimirdata}}
      \label{fig:figura}
    \end{figure}

    Em conclusão, admitiu-se uma flexibilização na execução de etapas mais
    próxima de uma iteração entre etapas e ou de todas etapas. Logo, prazos
    formais não foram admitidos pelo fato de que a aplicação provavelente não
    será empregada para uso concreto.

    % para planejamento e modelagem
    Com um processo de etapas e fluxo bem definido estabelecido segue-se,
    dentro deste modelo, para a etapa de planejamento e modelagem onde macro
    características significativas em impacto por todo o tempo de vida útil do
    \textit{software} serão inreversivelmente assumidas.

    Por conseguinte, o sistema, arquiteturalmente adota uma estruturação
    usualmente conhecida sob o título de \textit{Hexagonal Architecture}
    onde os objetos mais importantes da aplicação, denominados de objetos
    de domínio de negócio, encontram-se isolados dos demais objetos do ambiente
    referidos como \textit{outside world} "mundo externo" \cite{alistairHexagonal2005}.
    
    Por fim, a garantia de separação entre domínio e funcionalidades
    utilitárias dar-se-á por orientação da arquitetura sobrecitada que sinaliza
    o uso de classes pertecentes ao padrão de projeto \textit{Adapter} como
    meio de isolação.

    Em seguida, o sistema, de modo a atender os objetivos, possui classes que intermediam
    várias características indispensáveis ao funcionamento da aplicação.

    \begin{figure}[h]
      \centering
      \caption{ Interação entre Classes Nucleares do Domínio o \textit{Outside World}. }
      \includegraphics[scale=0.4]{lib/deployment-uml.png}
      \source{ O Autor {\imprimirdata}}
      \label{fig:figura}
    \end{figure}

    Uma vez que classes, em sistemas \ac{oop}, sempre possuem interelações
    com suas semelhantes na forma de herança ou composição ocorre que, em
    meios técnicos, discute-se acerca da necessidade de se priorizar uma sobre
    a outra. Não obstante, o objeto deste trabalho assume uma abordagem mista
    onde uma e outra são empregadas simultâneamente de acordo com o que se julga
    mais adequado e conveniente no contexto. 

    Em suma, de modo a prosseguir mais proximamente ao domínio de negócios
    caberá, mais apropiadamente, ao desenvolvimento, tratar acerca dos detalhes
    do domínio que inspirou a implementação deste experimento.
    

    %\input{textuais/capitulo_3/sub-topicos/QuestoesDePesquisa}
    %\input{textuais/capitulo_3/sub-topicos/PropostaMetodologica}
    %\input{textuais/capitulo_3/sub-topicos/Atividades}

\newpage
