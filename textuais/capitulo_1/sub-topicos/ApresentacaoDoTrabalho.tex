
% Esta seção trata a construção do documento, especificamente a disposição dos capítulos e uma introdução do que está sendo abordado nestes.

\subsection{\textbf{Apresentação do Trabalho}}
    \label{sec:apresentacao-trabalho}
    
    A introdução contextualiza brevemente o leitor à história
    da ciência da computação direcionando-o a um fato omnitemporal
    dessa ciência: as dependências e seus problemas potenciais. 

    A fundamentação teórica introduzirá o leitor aos fundamentos do
    paradigma de programação utilizado no projeto experimental e, em seguida,
    a seções de obras técnicas populares entre programadores juntando
    métodos conhecidos e padronizados com potencial para mitigar problemas
    oriundos da implementação do software.

    A metodologia apresentará como o objeto experimentado foi projetado
    em questão de planejamento, modelagem, construção, teste e entrega.

    Os resultados apontarão ocorrências e usos do que fora apresentado
    no referencial teórico. Foca-se na descrição do processo.

    A conclusão ou considerações finais, advindos da experiência obtida pela
    execução do projeto, apresentar-se-ão na forma de apontamentos dos beneficios
    e problemas oriundos da prática construtiva descrita em metodologia.



