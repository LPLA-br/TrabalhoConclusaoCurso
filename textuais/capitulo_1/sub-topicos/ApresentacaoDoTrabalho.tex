
% Esta seção trata a construção do documento, especificamente a disposição dos capítulos e uma introdução do que está sendo abordado nestes.

\subsection{\textbf{Apresentação do Trabalho}}
    \label{sec:apresentacao-trabalho}
    
    A introdução ora contexutalizada, brevemente levará o leitor à história da
    ciência da computação direcionando-o a um fato trágico dessa ciência: as
    dependências e seus problemas correlatos. 

    No tocante à fundamentação teórica, ela induzirá o leitor aos fundamentos
    do paradigma de programação utilizado no projeto experimental e, em
    seguida, a seções indiretas de obras técnicas populares entre programadores
    introdutoriamente compilando métodos conhecidos e padronizados com
    potencial para resolver ou mitigar os desafios correlatos às dependências.

    No que se refere a metodologia apresentada, buscaremos o procedimento padrão,
    no qual foram adotados ao projeto. Em outras palavras, detalhar-se-á: recursos
    empregados, uso do tempo, execução do processo, arquiteturas, padrões e
    princípios filosóficos. Todos aqueles de fato utilizados.

    Em se tratando do Desenvolvimento, apontamos que entrará em detalhes do
    domínio de negócio presente no cerne da aplicação objeto ao mesmo tempo em
    que descreverá o emprego real de métodos selecionados que propoem resolver
    ou mitigar a alta dependência externa.

    Diante do exposto esperamos que os resultados apontem fatos e descobertas
    ocorridas durante as atividades abrindo espaço para elaboração das
    conclusões.

    Neste sentido a concluimos que o respectivo estudo tragam reflexão a luz de
    obras do referencial teórico.



