
% Esta seção trata a construção do documento, especificamente a disposição dos capítulos e uma introdução do que está sendo abordado nestes.

\subsection{\textbf{Apresentação do Trabalho}}
    \label{sec:apresentacao-trabalho}
    
    A introdução contextualiza brevemente o leitor à história
    da ciência da computação direcionando-o a um fato trágico
    dessa ciência: as dependências e seus problemas correlatos. 

    A fundamentação teórica introduz o leitor aos fundamentos do paradigma de
    programação utilizado no projeto experimental e, em seguida, a seções
    indiretas de obras técnicas populares entre programadores introdutoriamente
    compilando métodos conhecidos e padronizados com potencial para resolver ou
    mitigar os desafios correlatos às dependências.

    A metodologia apresenta quais procedimentos padrões foram adotados no
    projeto. Em outras palavras, detalhar-se-á: recursos empregados, uso do
    tempo, execução do processo, arquiteturas, padrões e princípios filosóficos.
    Todos aqueles de fato utilizados.

    O Desenvolvimento entrará em detalhes do domínio de negócio presente no cerne
    da aplicação objeto ao mesmo tempo em que descreverá o emprego real de
    métodos que propoem resolver ou mitigar a alta dependência externa.

    Os resultados apontarão ocorrências e usos do que fora apresentado
    no referencial teórico. Foca-se na descrição do processo.

    A conclusão ou considerações finais, advindos da experiência obtida pela
    execução do projeto, apresentar-se-ão na forma de apontamentos dos beneficios
    e problemas oriundos da prática construtiva descrita em metodologia.



