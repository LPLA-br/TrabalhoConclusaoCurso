
% Esta seção trata a construção do documento, especificamente a disposição dos capítulos e uma introdução do que está sendo abordado nestes.

\subsection{\textbf{Apresentação do Trabalho}}
    \label{sec:apresentacao-trabalho}
    
    A introdução contextualiza brevemente o leitor à história
    da ciência da computação direcionando-o a um fato omnitemporal
    dessa ciência: as dependências e seus problemas potenciais. 

    A fundamentação teórica introduzirá o leitor aos fundamentos do
    paradigma de programação utilizado no projeto experimental e, em seguida,
    a seções de obras técnicas populares entre programadores juntando
    métodos conhecidos e padronizados com potencial para mitigar ou
    eliminar a dependência de um software de outro.

    A metodologia apresentará como o objeto experimentado foi concebido
    na forma de uma explanação não enlongada das etapas de:
    levantamento de requisitos e problema de negócio, modelagem,
    implemantação, testes, implantação. De forma a orientar
    como possíveis pesquisas posteriores organiza-se-ão.

    Os resultados

    A conclusão ou considerações finais, advindos da experiência obtida pela execução do projeto, 
    apresentar-se-ão na forma explanativa separativamente o
    que foi praticável do que não foi e apontando possíveis armadilhas
    das dependências escolhidas.



