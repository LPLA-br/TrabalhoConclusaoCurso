
% Esta seção trata a construção do documento, especificamente a disposição dos capítulos e uma introdução do que está sendo abordado nestes.

\subsection{\textbf{Apresentação do Trabalho}}
    \label{sec:apresentacao-trabalho}
    
    A introdução traz um contexto histórico do desenrolar da \ac{ti}
    em coexistência com a \ac{ia} e como isso pode estar criando armadilhas
    na forma de dependễncias.

    A Fundamentação teórica introduz o leitor a partes de obras que expoem
    métodos e técnicas que invertem ou reduzem dependências em
    sistemas.

    A metodologia explica a ordem como os processos de cada
    etapa do ciclo de vida do software foram realizados pelo
    pesquisador. Como foram levantados os requisitos, como foi
    modelada e projetada o esquema de classes, como foram implementadas
    e testadas, quais técnicas expostas na Fundamentação teórica foram, de fato,
    utilizadas em prol de se evitar as dependências.

    Os resultados mostram, por meio de diagramas e parágrafos explicativos,
    o \textit{status quo} da aplicação em uma versão minimamente utilizável
    exibindo o que foi praticável e impraticável para solução final.

    A conclusão definirá o nível de independência que foi possivel atingir
    deixando caminho para futuras pesquisas mais refinadas.

