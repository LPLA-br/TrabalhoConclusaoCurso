
% Esta seção contém os objetivos de sua pesquisa, contemplando o objetivo principal e as atividades para que este objetivo seja atingido.

\subsection{\textbf{Objetivos}}
    \label{sec:objetivos}

    O objetivo geral deste trabalho é determinar os efeitos causados
    pelo emprego de técnicas, padrões e ou filosofias que permitem
    que o \textit{software} seja mais independente de objetos, pacotes
    , frameworks e tecnologias externas e necessárias a sua utilidade existencial.
    Tudo isso sob orientação de obras conhecidas da industria de software.

    % sub-topicos/ContextualizacaoEProblema.tex deve possuir perguntas
    % aproximadamento correspondentes ao objetivos expostos abaixo.
    Seus objetivos especificos são:

    \begin{itemize}
      \item{Revisar obras que expoem métodos mitigacionais de acoplamento e dependência em sistemas.}
      \item{Compreender modelagem de sistemas com regras de negócios independentes de funcionalidades tereirizadas.}
      \item{Compreender benefícios e adversidades mais seus custos intrínsecos ao adotá-los.}
    \end{itemize}

