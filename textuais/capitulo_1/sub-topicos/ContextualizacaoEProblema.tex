
% Nesta seção devem ser introduzidos o ambiente (\textit{i.e.}Contexto) em que seu trabalho está inserido, além do problema que será abordado no seu trabalho.

\subsection{\textbf{Contextualização e Problema}}
    \label{subsec:contextualizacao-problema}

    A engenharia de software e ciência da computação reforçam a importância do
    emprego de boas práticas na criação de \textit{software} no contexto atual
    onde sistemas coordenam e aprimoram processos da sociedade humana.

    Entretanto, a predominância de movimentos pro ágil em empresas em conjunto com
    florecimento de modelos de inteligência artificial generativa nesta
    terceira década estão possibilitando o re-fortalecimento da crise do
    software.

    Configura-se-á como importante a produção de projetos que resgatem e
    enfatisem o progresso no estado da arte filosófico-técnico sobre
    metodologias técnicas pro qualidade no desenvolvimento de
    \textit{software}. Questionam-se:

    \begin{flushleft}
    Sistemas independentes e dasacoplados de suas dependências são possíveis ?
    Quais beneficios e adversidades recaem sobre projetos com esta orientação ?
    \end{flushleft}


