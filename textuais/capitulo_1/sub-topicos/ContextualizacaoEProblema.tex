
% Nesta seção devem ser introduzidos o ambiente (\textit{i.e.}Contexto) em que seu trabalho está inserido, além do problema que será abordado no seu trabalho.

\subsection{\textbf{Contextualização e Problema}}
    \label{subsec:contextualizacao-problema}

    A engenharia de software e ciência da computação reforçam a importância do
    emprego de boas práticas na criação de sistemas no contexto moderno onde
    eles coordenam e aprimoram processos informacionais da sociedade humana.

    Entretanto, o uso de metodologias ágeis como bala de prata para geração de
    valor em tempo recorde em conjunto com florecimento de modelos de
    inteligência artificial generativa, supostamente capazes de substituir o
    raciocínio humano sem problemas e com baixo custo, nesta terceira década
    estão abrindo caminho para o renascimento da crise do software sobre uma
    nova face.

    Vê-se o uso incauteloso das mais diversas ferramentas nas formas
    de bibliotecas e \textit{frameworks} que fornecem meios rápidos de agregar
    funcionalidade às aplicações. Todavia, sem mínimo compromisso de atenção com
    os problemas futuros que a dependência pode infligir sobre projetos cujo
    tempo de vida útil pode ser importante.


    \begin{flushleft}
    Com vistas ao contexto e seu problema, formula-se duas questões:

    Sistemas independentes e desacoplados de suas dependências são possíveis ?

    Quais são os beneficios e problemas que recaem sobre projetos que assumem um caminho de independência estrita ?
    \end{flushleft}


