
% Nesta seção devem ser introduzidos o ambiente (\textit{i.e.}Contexto) em que seu trabalho está inserido, além do problema que será abordado no seu trabalho.

\subsection{\textbf{Contextualização e Problema}}
    \label{subsec:contextualizacao-problema}

    \textit{Nesta seção devem ser introduzidos o ambiente (\textit{i.e.}Contexto) em que seu trabalho
    está inserido, além do problema que será abordado no seu trabalho.}

    Além de uma gestão ágil, onde os programadores têm liberdade de se adaptarem e
    se auto organizarem de forma a proverem aos clientes a geração de valor, há
    a necessidade de reforçar os princípios arquiteto-implementacionais que auxiliam
    a garantir a qualidade do software.
      Entretanto vê-se que, sem empenho na aplicação de padronizações provadas pelo
    tempo, o software moderno têm sua sofisticação implementacional preventiva negligenciada em
    nome da agilidade a todo custo.
      A terceirização indiscriminada de boa parte da responsabilidade de pensar e agir
    acerca do ambiente de negócios e técnico para a \ac{ia} gera uma crise invisível que
    pode tomar dimensôes nunca antes vistas, proximas, em semelhança, com a crise do software
    de 1960 a 1980.

