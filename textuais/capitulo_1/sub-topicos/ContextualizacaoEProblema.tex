
% Nesta seção devem ser introduzidos o ambiente (\textit{i.e.}Contexto) em que seu trabalho está inserido, além do problema que será abordado no seu trabalho.

\subsection{\textbf{Contextualização e Problema}}
    \label{subsec:contextualizacao-problema}

    A engenharia de software,bem como a ciência da computação, reforçam a
    importância do emprego de boas práticas na criação de sistemas no contexto
    moderno, onde eles coordenam e aprimoram processos informacionais da
    sociedade humana.

    Entretanto, a tendência dos negócios de tecnologia a adotar metodologias
    ágeis como solução ótima em todos contextos adjunta ao surgimento recente da
    inteligência artificial nesta terceira década estarão abrindo caminho para
    o re-aumento em aplitude da crise do software sob novas características.

    Dentre os problemas nascidos desta nova crise, vê-se o uso incauteloso das
    mais diversas ferramentas existentes nas formas de bibliotecas e
    \textit{frameworks}. A pricípio, tais ferramentas úteis agregação rápida
    de funcionalidades nas aplicações agora consolidam-se como fator propulsor
    da queda de qualidade e obsolescência em sistemas com anos de manutenção.

    \begin{flushleft}
    Com vistas ao contexto e seu problema, formula-se duas questões:

    Sistemas independentes e desacoplados de suas dependências são possíveis ?

    Quais são os beneficios e problemas que recaem sobre projetos que assumem um caminho de independência estrita ?
    \end{flushleft}


