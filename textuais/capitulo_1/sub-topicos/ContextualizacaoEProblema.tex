
% Nesta seção devem ser introduzidos o ambiente (\textit{i.e.}Contexto) em que seu trabalho está inserido, além do problema que será abordado no seu trabalho.

\subsection{\textbf{Contextualização e Problema}}
    \label{subsec:contextualizacao-problema}

    O desenvolvimento de software atualmente é impelido, na maioria das organizações,
    por times ágeis auto gerenciados auxiliados pela maravilha desta terceira década:
    A inteligência artificial generativa. Em tal contexto de auxilio constante algoritmos
    não deterministicos pode-se acompanhar o desenrolar de um cenário novo onde a
    qualidade dos sistemas desenvolvidos são, como nunca visto antes, fortemente
    influenciadas tais ferramentas levando a um desfecho ainda incerto.

    Tem-se como certo que problemas de qualidade implementacional originadas pela terceirização
    indiscriminada de boa parte da responsabilidade de pensar e agir em desenvolvendo um software
    à \ac{ia} podem levar a uma crise relativamente semelhante a que acontecera
    durante a crise do software das décadas de 1960 e 1980.

    Nunca foi tão importante como agora a necessidade pela plurificação de trabalhos que
    visem reforçar a exemplificação do emprego de conhecimentos filosófico-técnicos que
    interfiram positivamente na entrega de valor do software hodiernamente.

    Tendo em vista a realidade corrente, determina-se como problema cerne desta pesquisa: Quais
    são os efeitos, sejam eles adversos ou benéficos, de se implementar software
    cuja suas regras de negócio sejam totalmente independentes de bibliotecas e frameworks
    providas por empresas ou grupos colaborativos de programadores ?

    

