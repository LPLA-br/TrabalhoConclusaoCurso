
%%%%%%%%%%%%%%%%%%%%%%%%%%%%%%%%%%%%%%
%%            Introdução            %%
%%%%%%%%%%%%%%%%%%%%%%%%%%%%%%%%%%%%%%

% Neste capítulo devem ser apresentados o contexto do seu trabalho e o problema que deseja abordar, bem como os objetivos que deseja alcançar e a justificativa para o estudo.

\section{\textbf{INTRODUÇÃO}}
    \label{sec:introdução}
    % Tópicos propostos, para remover algum destes tópicos
    %comente a linha referente ao tópico que deseja remover
    Neste capítulo devem ser apresentados o contexto do seu trabalho e o problema que
    deseja abordar, bem como os objetivos que deseja alcançar e a justificativa para o estudo.
    
    % Contextualização e problema
    No século XXI o software é o mais importante produto da industria 4.0 e 5.0 .
    Vários setores da economia conteporânea apoiam sua logistica de produção de produtos e 
    provimentos de serviços por meio de tais ferramentas (expandir).

    Nestes primeiros anos da década de 2020 viu-se, não somente no
    mercado nacional, o pico na demanda por profissionais com
    competências para construir soluções em \ac{ti}, demanda esta justificada pelos
    problemas sanitários do período pandêmico que limitaram o contato humano e favoreceram a expansão
    da infraestrutura digital em campos como a educação, comunicações,
    comércio digital e outros \cite{vieira2024impactos} e \cite{carreira2023}.
    
    Entretanto, o que se seguiu foi um cenário de demissões em massa e uma aparente
    queda de demanda por profissionais, o periódico  \citeonline{carreira2023} aponta
    como principais motivos o pós pandemia, que trouxe redução de demanda justificada
    pelos novos níveis de receitas das companhias de tecnologia e a recessão global
    que dá sinais de crise a bastante tempo.

    Atualmente, 50\% profissionais apontam salários menores e
    jornadas mais exaustivas enquanto 45\% empresas, demandantes de habilidades mais
    concretamente estabelecidas, enfretam dificuldades para contratar profissionais
    realmente qualificados \cite{cnn2024}.

    A dificuldade e necessidade em conseguir profissionais qualificados apontam para
    um cenario onde a entrega de valor é uma prioridade maior em tempos de
    estabilização do mercado e garanti-la através de profissionais mais capazes
    em evitar os efeitos do débito técnico, comumente associados à falta de experiência,
    é nuclear \cite[p.~131 et all.]{beltrao2020}. Além disto o advento da \ac{ia} possibilitou a automação parcial ou
    completa de processos simples que antes eram exercidos por profissionais menos
    experientes como bem demonstrado por \citeonline{batista2023ia}: Ele
    apontara capacidades de, com prompts simples, construir códigos de calculadora
    e sua interface funcionais, estilização básica consistente e até mesmo um quadro de desenhos
    reinicialiável.

    O autor \citeonline[p.~131 et all.]{beltrao2020} introduz o débito técnico como
    sendo o foco em benefícios de curto prazo acaba criando problemas enormes na manutenibilidade
    de um sistema implicando em uma realidade de custos maiores e, em casos extremos,
    de re-implementações totais de novos sistemas. O autor \cite{?} mostra que X\% dos sistemas
    desenvolvidos em solo nacional enfretam ou enfrentarão problemas oriundos do
    excercício imprudente de implementações sem planejamento ou padrões.
    
    
% Nesta seção devem ser introduzidos o ambiente (\textit{i.e.}Contexto) em que seu trabalho está inserido, além do problema que será abordado no seu trabalho.

\subsection{\textbf{Contextualização e Problema}}
    \label{subsec:contextualizacao-problema}

    O desenvolvimento de software atualmente é impelido, na maioria das organizações,
    por times ágeis auto gerenciados auxiliados pela maravilha desta terceira década:
    A inteligência artificial generativa. Em tal contexto de auxilio constante algoritmos
    não deterministicos pode-se acompanhar o desenrolar de um cenário novo onde a
    qualidade dos sistemas desenvolvidos são, como nunca visto antes, fortemente
    influenciadas tais ferramentas levando a um desfecho ainda incerto.

    Tem-se como certo que problemas de qualidade implementacional originadas pela terceirização
    indiscriminada de boa parte da responsabilidade de pensar e agir em desenvolvendo um software
    à \ac{ia} podem levar a uma crise relativamente semelhante a que acontecera
    durante a crise do software das décadas de 1960 e 1980.

    Nunca foi tão importante como agora a necessidade pela plurificação de trabalhos que
    visem reforçar a exemplificação do emprego de conhecimentos filosófico-técnicos que
    interfiram positivamente na entrega de valor do software hodiernamente.

    Tendo em vista a realidade corrente, determina-se como problema cerne desta pesquisa: Quais
    são os efeitos, sejam eles adversos ou benéficos, de se implementar software
    cuja suas regras de negócio sejam totalmente independentes de bibliotecas e frameworks
    providas por empresas ou grupos colaborativos de programadores ?

    


    % Objetivos (geral e específicos)
    
% Esta seção contém os objetivos de sua pesquisa, contemplando o objetivo principal e as atividades para que este objetivo seja atingido.

\subsection{\textbf{Objetivos}}
    \label{sec:objetivos}
    
    \textit{Esta seção contém os objetivos de sua pesquisa, contemplando o objetivo principal
    e as atividades para que este objetivo seja atingido.}

    Este trabalho tem como objetivo principal pertencer a um conjunto de trabalhos, 
    escritos por outros autores, voltados à exemplificação do emprego total ou quase
    total de padrões arquiteturais sob as restrições de uma tecnologia e sua(s) sub ferramenta(s)
    comercialmente bem sucedidas e largamente utilizadas.
      Sua natureza é aplicada em um projeto real de código fonte aberto e seus objetivos são
    alcaçados por meio da exploração de um cenário cujo o desfecho se orienta pelos axiomas,
    postulados e princípios construidos ao longo de décadas de experiência de autores/programadores
    mundo afora expostos em obras literárias recentes.
      

    % Delimitação do estudo
    
% Nesta seção deve ser apresentado o problema abordado no estudo de forma a especificá-lo dentro do contexto em que está inserido.

\subsection{\textbf{Delimitação do Estudo}}
    \label{sec:delimitacao-estudo}
    
    Nesta seção deve ser apresentado o problema abordado no estudo apresentando exatamente o que será considerado para o mesmo e deixando claro o que não será levado em conta. Considerando um trabalho que fará um estudo comparativo entre algoritmos a partir do desempenho em bases de dados.
    
    Neste trabalho serão observados apenas X técnicas, são elas, Técnica1, Técnica2, ..., TécnicaN, pois são técnicas utilizadas para classificação de dados em bancos de dados não relacionais. Dentro delas, Y parâmetros são abordados e Z não são pelos motivos AC/DC.
    
    % de forma a especificá-lo dentro do contexto em que está inserido.

    % Justificativa
    
% Esta seção trata os motivos pelos quais seu trabalho é relevante

\subsection{\textbf{Justificativa}}
    \label{sec:justificativa}
    
    Este trabalho justifica-se como sendo, no âmbito acadêmico, um esforço experimental
    em prol de obter conclusões restritas sobre uma tecnlogia acerca de um tópico de
    grande interesse na engenharia de software: dependências.

    Quanto ao âmbito pessoal/profissional, este trabalho é uma forma de aprimorar os conjuntos
    de conhecimentos técnico-arquiteturais do docente, permitindo-lhe um acesso mais seguro
    ao mercado de trabalho de TI.

    Trata-se do emprego do conhecimento pré-existente sobre técnicas implementacionais
    em um sistema real de modo a determinar os limites da implementação de aplicações
    com baixo nível de dependência entre a solução e as ferramentas da solução.


    %\begin{enumerate}
    %    \item O que a comunidade acadêmica irá ganhar com seu trabalho?;
    %    \item Qual a razão do seu trabalho ser desenvolvido?;
    %\end{enumerate}
    

    % Apresentação do trabalho
    
% Esta seção trata a construção do documento, especificamente a disposição dos capítulos e uma introdução do que está sendo abordado nestes.

\subsection{\textbf{Apresentação do Trabalho}}
    \label{sec:apresentacao-trabalho}
    
    Esta seção trata a construção do documento, especificamente a disposição dos capítulos e uma introdução do que está sendo abordado nestes.


% Nesta seção devem ser introduzidos o ambiente (\textit{i.e.}Contexto) em que seu trabalho está inserido, além do problema que será abordado no seu trabalho.
\newpage
