
%%%%%%%%%%%%%%%%%%%%%%%%%%%%%%%%%%%%%%
%%            Introdução            %%
%%%%%%%%%%%%%%%%%%%%%%%%%%%%%%%%%%%%%%

% Neste capítulo devem ser apresentados o contexto do seu trabalho e o problema que deseja abordar, bem como os objetivos que deseja alcançar e a justificativa para o estudo.

\section{\textbf{INTRODUÇÃO}}
    \label{sec:introdução}
    % Tópicos propostos, para remover algum destes tópicos
    %comente a linha referente ao tópico que deseja remover
    Neste capítulo devem ser apresentados o contexto do seu trabalho e o problema que
    deseja abordar, bem como os objetivos que deseja alcançar e a justificativa para o estudo.
    
    % Contextualização e problema
    No século XXI o software é o mais importante produto da industria 4.0 e 5.0 .
    Vários setores da economia conteporânea apoiam sua logistica de produção de produtos e 
    provimentos de serviços por meio de tais ferramentas (expandir).

    Nestes primeiros anos da década de 2020 viu-se, não somente no
    mercado nacional, o pico na demanda por profissionais com
    competências para construir soluções em \ac{ti}, demanda esta justificada pelos
    problemas sanitários do período pandêmico que limitaram o contato humano e favoreceram a expansão
    da infraestrutura digital em campos como a educação, comunicações,
    comércio digital e outros \cite{vieira2024impactos} e \cite{carreira2023}.
    
    Entretanto, o que se seguiu foi um cenário de demissões em massa e uma aparente
    queda de demanda por profissionais, o periódico  \citeonline{carreira2023} aponta
    como principais motivos o pós pandemia, que trouxe redução de demanda justificada
    pelos novos níveis de receitas das companhias de tecnologia e a recessão global
    que dá sinais de crise a bastante tempo.

    Atualmente, 50\% profissionais apontam salários menores e
    jornadas mais exaustivas enquanto 45\% empresas, demandantes de habilidades mais
    concretamente estabelecidas, enfretam dificuldades para contratar profissionais
    realmente qualificados \cite{cnn2024}.

    A dificuldade e necessidade em conseguir profissionais qualificados apontam para
    um cenario onde a entrega de valor é uma prioridade maior em tempos de
    estabilização do mercado e garanti-la através de profissionais mais capazes
    em evitar os efeitos do débito técnico, comumente associados à falta de experiência,
    é nuclear \cite[p.~131 et all.]{beltrao2020}. Além disto o advento da \ac{ia} possibilitou a automação parcial ou
    completa de processos simples que antes eram exercidos por profissionais menos
    experientes como bem demonstrado por \citeonline{batista2023ia}: Ele
    apontara capacidades de, com prompts simples, construir códigos de calculadora
    e sua interface funcionais, estilização básica consistente e até mesmo um quadro de desenhos
    reinicialiável.

    O autor \citeonline[p.~131 et all.]{beltrao2020} introduz o débito técnico como
    sendo o foco em benefícios de curto prazo acaba criando problemas enormes na manutenibilidade
    de um sistema implicando em uma realidade de custos maiores e, em casos extremos,
    de re-implementações totais de novos sistemas. O autor \cite{?} mostra que X\% dos sistemas
    desenvolvidos em solo nacional enfretam ou enfrentarão problemas oriundos do
    excercício imprudente de implementações sem planejamento ou padrões.
    
    
% Nesta seção devem ser introduzidos o ambiente (\textit{i.e.}Contexto) em que seu trabalho está inserido, além do problema que será abordado no seu trabalho.

\subsection{\textbf{Contextualização e Problema}}
    \label{subsec:contextualizacao-problema}

    \textit{Nesta seção devem ser introduzidos o ambiente (\textit{i.e.}Contexto) em que seu trabalho
    está inserido, além do problema que será abordado no seu trabalho.}

    Além de uma gestão ágil, onde os programadores têm liberdade de se adaptarem e
    se auto organizarem de forma a proverem aos clientes a geração de valor, há
    a necessidade de reforçar os princípios arquiteto-implementacionais que auxiliam
    a garantir a qualidade do software.
      Entretanto vê-se que, sem empenho na aplicação de padronizações provadas pelo
    tempo, o software moderno têm sua sofisticação implementacional preventiva negligenciada em
    nome da agilidade a todo custo.
      A terceirização indiscriminada de boa parte da responsabilidade de pensar e agir
    acerca do ambiente de negócios e técnico para a \ac{ia} gera uma crise invisível que
    pode tomar dimensôes nunca antes vistas, proximas, em semelhança, com a crise do software
    de 1960 a 1980.


    % Objetivos (geral e específicos)
    
% Esta seção contém os objetivos de sua pesquisa, contemplando o objetivo principal e as atividades para que este objetivo seja atingido.

\subsection{\textbf{Objetivos}}
    \label{sec:objetivos}
    
    \textit{Esta seção contém os objetivos de sua pesquisa, contemplando o objetivo principal
    e as atividades para que este objetivo seja atingido.}

    Este trabalho tem como objetivo principal pertencer a um conjunto de trabalhos, 
    escritos por outros autores, voltados à exemplificação do emprego total ou quase
    total de padrões arquiteturais sob as restrições de uma tecnologia e sua(s) sub ferramenta(s)
    comercialmente bem sucedidas e largamente utilizadas.
      Sua natureza é aplicada em um projeto real de código fonte aberto e seus objetivos são
    alcaçados por meio da exploração de um cenário cujo o desfecho se orienta pelos axiomas,
    postulados e princípios construidos ao longo de décadas de experiência de autores/programadores
    mundo afora expostos em obras literárias recentes.
      

    % Delimitação do estudo
    \input{textuais/capitulo_1/sub-topicos/DelimitacaoDoEstudo}
    % Justificativa
    
% Esta seção trata os motivos pelos quais seu trabalho é relevante

\subsection{\textbf{Justificativa}}
    \label{sec:justificativa}
    
    Este trabalho justifica-se como sendo, no âmbito acadêmico, um esforço experimental
    para incremento no número de obras que apresentam o emprego prático-experimental
    de metodologias, técnicas e ou filosofias desenvolvidas para solucionar ou mitigar
    problemas que assolam o desenvolvimento software.

    Quanto ao âmbito pessoal/profissional, este trabalho é uma oportunidade de
    aprimorar os conjuntos de conhecimentos técnico-arquiteturais do docente,
    permitindo-lhe um acesso mais seguro ao mercado de trabalho de TI em
    implementações comercialmente viáveis.

    Trata-se do emprego real do conhecimento pré-existente em prol de obter
    conclusões sobre as limitações e possibilidades proporcionados pela prática.

    %\begin{enumerate}
    %    \item O que a comunidade acadêmica irá ganhar com seu trabalho?;
    %    \item Qual a razão do seu trabalho ser desenvolvido?;
    %\end{enumerate}
    

    % Apresentação do trabalho
    
% Esta seção trata a construção do documento, especificamente a disposição dos capítulos e uma introdução do que está sendo abordado nestes.

\subsection{\textbf{Apresentação do Trabalho}}
    \label{sec:apresentacao-trabalho}
    
    A introdução contextualiza brevemente o leitor à história
    da ciência da computação direcionando-o a um fato omnitemporal
    dessa ciência: as dependências e seus problemas potenciais. 

    A fundamentação teórica introduzirá o leitor aos fundamentos do
    paradigma de programação utilizado no projeto experimental e, em seguida,
    a seções de obras técnicas populares entre programadores juntando
    métodos conhecidos e padronizados com potencial para mitigar ou
    eliminar a dependência de um software de outro.

    A metodologia apresentará como o objeto experimentado foi concebido
    na forma de uma explanação não enlongada das etapas de:
    levantamento de requisitos e problema de negócio, modelagem,
    implemantação, testes, implantação. De forma a orientar
    como possíveis pesquisas posteriores organiza-se-ão.

    Os resultados

    A conclusão ou considerações finais, advindos da experiência obtida pela execução do projeto, 
    apresentar-se-ão na forma explanativa separativamente o
    que foi praticável do que não foi e apontando possíveis armadilhas
    das dependências escolhidas.





% Nesta seção devem ser introduzidos o ambiente (\textit{i.e.}Contexto) em que seu trabalho está inserido, além do problema que será abordado no seu trabalho.
\newpage
