
%%%%%%%%%%%%%%%%%%%%%%%%%%%%%%%%%%%%%%
%%            Introdução            %%
%%%%%%%%%%%%%%%%%%%%%%%%%%%%%%%%%%%%%%

% Neste capítulo devem ser apresentados o contexto do seu trabalho e o problema que
% deseja abordar, bem como os objetivos que deseja alcançar e a justificativa para o estudo.

\section{\textbf{INTRODUÇÃO}}
    \label{sec:introdução}
    % Tópicos propostos, para remover algum destes tópicos
    %comente a linha referente ao tópico que deseja remover

    %\textit{Neste capítulo devem ser apresentados o contexto do seu trabalho e o problema que
    %deseja abordar, bem como os objetivos que deseja alcançar e a justificativa para o estudo.}

    % introdução - parágrafos - novo planejamento

    % I. Software no início da computação
    Os computadores eletrônicos e o \textit{software} emergiram como ferramenta
    de importância notória durante os eventos finais da segunda grande guerra.
    Seu emprego possibilitara computação de trajetórias balisticas,
    propriedades das detonações atômicas e quebra de de criptografia inimiga
    \cite[~p.4]{dodig2001history}.
    
    Poucos anos após o fim do conflito, na década de 1950, houve o
    desenvolvimento contínuo não somente da tecnologia física como também da
    lógica "\textit{software}". Programas, antes escritos em binário, agora
    podiam ser criados em linguagens cada vez mais próximas da linguagem humana
    \cite[~p.5]{dodig2001history}.

    Na década 1960 a Ciência da Computação consolidara-se, de fato, como uma
    ciência formal e de importância notoria para as décadas futuras. Sua
    importância não mais retringia-se acadêmico-militar mas, agora,
    influenciava positivamente aos grandes negócios, instituições públicas e
    universidades \cite{newell1967what}.

    Conteporâneamente via-se um linear, quase exponencial, aumento do poder
    computacional ao mesmo tempo em que diminuia-se as dimensões físicas dos
    computadores. De acordo com \citeonline{almeidaevoluccao} o fundador da
    Intel, Gordon Moore, publicara na revista \textit{Electronics Magazine},
    um artigo, onde previra aumento em dobro do poder computacional a cada 18
    meses fato esse que possibilitara o desenvolvimento de aplicações escalantes

    % II. Software, complexidade crescente das soluções
    Entretanto, nenhum avanço é insento de problemas e crises. Durante a
    décadas de 1960 e 1970 o grande aumento na necessidade por
    \textit{software} por organizações comerciais, públicas e de ensino
    impeliram o agravamento da chamada "Crise do Software". O cenário teve como
    característica principal, além do aumento constante do poder computacional
    \cite[~p.3]{dijkstra1972humble}, a ineficácia de processos e técnicas
    empregadas na implementação do \textit{software}, isso em um cenário onde
    aplicações cada vez mais complexas eram exigidas como apresentado em resumo
    por \cite{softwareCrisis2}.

    % TENTAR ACESSAR LOGADO NA EDUROAM DO IFRN PARA PRODUÇÃO DO TEXTO
    % Science Direct: Solution in software crisis Author links open overlay panelAntti Auer, Mikko Levanto, Ari Okkonen, Jyrki Okkonen
    % Software Crisis 2.0 Brian Fitzgerald
 
    % III. Crise do software de 1960,70,80
    O cenário problemático de outrora trouxera efeitos indesejáveis em todos os
    âmbitos do empreendimento de se escrever \textit{software}. Complexidade
    crescia enquanto a imaturidade técnico-metodologica firmava-se como fator
    propulsor da crise \cite[~p.14]{brooks1995mythical}. Apresentavam-se como
    consequências dos maus processos O(A):

    \begin{itemize}
      \item{ Estouro de prazos previamente estipulados. }
      \item{ Aumento de custo muito além do planejado. }
      \item{ Agravamento da inconsistência entre o que era exigido e o que era provido. }
      \item{ Expansão da ingerenciabilidade de projetos em pouco tempo de manutenção. }
      \item{ Queda de qualidade rápida da base de código fonte. }
    \end{itemize}

    % IV. Mitigação da crise e progresso até os dias atuais.
    Apesar da crise em si nunca ter sido absolutamente resolvida de fato, esforços
    oriundos da experiência de vários profissionais e pesquisadores competentes
    resultaram na consolidação de obras técnicas com orientações fundamentais que
    propõem soluções efetivas, como: \textit{The Mytical Man-Month, Design Patterns: Elements
    of Reusable Object-Oriented Software, Structure Programming - Dijkstra, Software
    Engineering Conference NATO Science Committee} dentre outras.

    % IV. Crise atual do software. - A dependência
    Em meio a vários problemas enfrentados pela ciência da computação e engenharia de
    software há o da dependência como possível ocasionadora do alto
    acoplamento em aplicações modernas. O software, como solução, surge, da
    mesma maneira que a pesquisa científica, alicerçando-se sobre outros
    \textit{softwares} desenvolvidos anteriormente por outros programadores.
    Tais alicerces denominaram-se: módulos, bibliotecas e frameworks \cite{IEEEGlossary}.

    Os artefatos alicerçantes de \textit{software} nascem com objetivo de
    prover soluções genéricas que acabam, por vezes, ocasionando situações em que
    lógica de negócio, principal parte da aplicação em desenvolvimento,
    mistura-se com as soluções por elas providas implicando, assim, em um acoplamento
    potêncialmente perigoso e tóxico às manutenções, extensões e testes
    \cite{evans2004domain, uncleBob2017}.

    % V. Conclusão da introdução - Pode-se evitar a dependência ?
    O problema que se apresenta possui inúmeras análises de inúmeros
    autores que buscam extender o estado da arte no âmbito do \textit{software}
    e seus alicerces dependidos. Buscam, eles, padrões que equilibrem
    os antônimos omnipresentes na tecnologia da informação: dependência/independência.
    
    Tais análises partem das experiências de seus autores e devem, no esforço acadêmico,
    serem experimentadas concomitantemente com ambientes possuidores de diversas
    variáveis mutáveis. Avaliar-se-á a aplicabilidade por meio de projetos
    observantes do que apontou-se como problema aplicando-lhe uma ou várias soluções
    propostas.

    
% Nesta seção devem ser introduzidos o ambiente (\textit{i.e.}Contexto) em que seu trabalho está inserido, além do problema que será abordado no seu trabalho.

\subsection{\textbf{Contextualização e Problema}}
    \label{subsec:contextualizacao-problema}

    \textit{Nesta seção devem ser introduzidos o ambiente (\textit{i.e.}Contexto) em que seu trabalho
    está inserido, além do problema que será abordado no seu trabalho.}

    Além de uma gestão ágil, onde os programadores têm liberdade de se adaptarem e
    se auto organizarem de forma a proverem aos clientes a geração de valor, há
    a necessidade de reforçar os princípios arquiteto-implementacionais que auxiliam
    a garantir a qualidade do software.
      Entretanto vê-se que, sem empenho na aplicação de padronizações provadas pelo
    tempo, o software moderno têm sua sofisticação implementacional preventiva negligenciada em
    nome da agilidade a todo custo.
      A terceirização indiscriminada de boa parte da responsabilidade de pensar e agir
    acerca do ambiente de negócios e técnico para a \ac{ia} gera uma crise invisível que
    pode tomar dimensôes nunca antes vistas, proximas, em semelhança, com a crise do software
    de 1960 a 1980.


    % Objetivos (geral e específicos)
    
% Esta seção contém os objetivos de sua pesquisa, contemplando o objetivo principal e as atividades para que este objetivo seja atingido.

\subsection{\textbf{Objetivos}}
    \label{sec:objetivos}
    
    \textit{Esta seção contém os objetivos de sua pesquisa, contemplando o objetivo principal
    e as atividades para que este objetivo seja atingido.}

    Este trabalho tem como objetivo principal pertencer a um conjunto de trabalhos, 
    escritos por outros autores, voltados à exemplificação do emprego total ou quase
    total de padrões arquiteturais sob as restrições de uma tecnologia e sua(s) sub ferramenta(s)
    comercialmente bem sucedidas e largamente utilizadas.
      Sua natureza é aplicada em um projeto real de código fonte aberto e seus objetivos são
    alcaçados por meio da exploração de um cenário cujo o desfecho se orienta pelos axiomas,
    postulados e princípios construidos ao longo de décadas de experiência de autores/programadores
    mundo afora expostos em obras literárias recentes.
      

    % Delimitação do estudo
    %\input{textuais/capitulo_1/sub-topicos/DelimitacaoDoEstudo}
    % Justificativa
    
% Esta seção trata os motivos pelos quais seu trabalho é relevante

\subsection{\textbf{Justificativa}}
    \label{sec:justificativa}
    
    Este trabalho justifica-se como sendo, no âmbito acadêmico, um esforço experimental
    para incremento no número de obras que apresentam o emprego prático-experimental
    de metodologias, técnicas e ou filosofias desenvolvidas para solucionar ou mitigar
    problemas que assolam o desenvolvimento software.

    Quanto ao âmbito pessoal/profissional, este trabalho é uma oportunidade de
    aprimorar os conjuntos de conhecimentos técnico-arquiteturais do docente,
    permitindo-lhe um acesso mais seguro ao mercado de trabalho de TI em
    implementações comercialmente viáveis.

    Trata-se do emprego real do conhecimento pré-existente em prol de obter
    conclusões sobre as limitações e possibilidades proporcionados pela prática.

    %\begin{enumerate}
    %    \item O que a comunidade acadêmica irá ganhar com seu trabalho?;
    %    \item Qual a razão do seu trabalho ser desenvolvido?;
    %\end{enumerate}
    

    % Apresentação do trabalho
    
% Esta seção trata a construção do documento, especificamente a disposição dos capítulos e uma introdução do que está sendo abordado nestes.

\subsection{\textbf{Apresentação do Trabalho}}
    \label{sec:apresentacao-trabalho}
    
    A introdução contextualiza brevemente o leitor à história
    da ciência da computação direcionando-o a um fato omnitemporal
    dessa ciência: as dependências e seus problemas potenciais. 

    A fundamentação teórica introduzirá o leitor aos fundamentos do
    paradigma de programação utilizado no projeto experimental e, em seguida,
    a seções de obras técnicas populares entre programadores juntando
    métodos conhecidos e padronizados com potencial para mitigar ou
    eliminar a dependência de um software de outro.

    A metodologia apresentará como o objeto experimentado foi concebido
    na forma de uma explanação não enlongada das etapas de:
    levantamento de requisitos e problema de negócio, modelagem,
    implemantação, testes, implantação. De forma a orientar
    como possíveis pesquisas posteriores organiza-se-ão.

    Os resultados

    A conclusão ou considerações finais, advindos da experiência obtida pela execução do projeto, 
    apresentar-se-ão na forma explanativa separativamente o
    que foi praticável do que não foi e apontando possíveis armadilhas
    das dependências escolhidas.





% Nesta seção devem ser introduzidos o ambiente (\textit{i.e.}Contexto) em que seu trabalho está inserido, além do problema que será abordado no seu trabalho.
\newpage
