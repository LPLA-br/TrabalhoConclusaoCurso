
%%%%%%%%%%%%%%%%%%%%%%%%%%%%%%%%%%%%%%
%%            Introdução            %%
%%%%%%%%%%%%%%%%%%%%%%%%%%%%%%%%%%%%%%

% Neste capítulo devem ser apresentados o contexto do seu trabalho e o problema que
% deseja abordar, bem como os objetivos que deseja alcançar e a justificativa para o estudo.

\section{\textbf{INTRODUÇÃO}}
    \label{sec:introdução}
    % Tópicos propostos, para remover algum destes tópicos
    %comente a linha referente ao tópico que deseja remover

    %\textit{Neste capítulo devem ser apresentados o contexto do seu trabalho e o problema que
    %deseja abordar, bem como os objetivos que deseja alcançar e a justificativa para o estudo.}

    % introdução - parágrafos - novo planejamento

    % I. Software no início da computação
    A computadores eletrônicos e o software surgiram como ferramenta de importância
    reconhecida notoriamente durante os eventos finais da segunda grande guerra.
    Inicialmente tinha-se como foco utilitário a quebra de criptografia
    da comunicação inimiga e o cálculo da tragetória de mísseis balisticos.
    
    Poucos anos após o fim dos conflitos armados houve o desenvolvimento contínuo
    não somente da tecnologia física como também da lógica (\textit{software}).
    Programas, antes escritos em binário agora podiam ser criados em linguagens
    próximas da linguagem humana.

    O trabalho da cientista Grace Murray Hopper, década de 1950,
    concebeu a ideia do compilador, programa capaz de transcrever/traduzir uma
    linguagem lógica mais abstrata e inteligível ao homem para o código de máquina
    executável pelo aparato físico computacional.

    Na década 1960 a Ciência da Computação consolidara-se, de fato, como uma ciência formal
    e de importância notoria para as décadas futuras. Sua importância não mais
    era militar mas também civil em áreas como negócios, administração e demais áreas das
    ciências exatas e da natureza.

    Via-se um linear e vertiginoso aumento do poder computacional ao mesmo tempo em
    que diminuia-se as dimensões dos computadores. Gordon Moore, fundador da Intel,
    publicara na revista \textit{ Electronics Magazine }, em 1965, um artigo, onde previra
    o dobro do aumento do poder computacional a cada 18 meses.
    %@article{almeidaevoluccao, title={Evolu{\c{c}}{\~a}o dos Processadores}, author={Almeida, Rafael Bruno}, journal={Evolu{\c{c}}{\~a}o dos processadores} }

    % II. Software, complexidade crescente das soluções
    Entretanto, como bem compreendido na historiologia tecnologica humana, nenhum
    avanço é insento de problemas e crises. Durante a décadas de 1960 e 1970 houve um grande aumento
    na necessidade por software por parte de organizações governamentais e privadas. Todavia,
    processos e técnicas praticadas no desenvolvimento conhecidas na época não eram eficazes
    o suficiente em atender as necessidades por soluções de complexidade crescente.
    % Página 5, Dijkstra, The humble programmer
 
    % III. Crise do software de 1960,70,80
    O cenário de crise possuia como principais características: Complexidade crescente das
    soluções exigidas e imaturidade técnico-metodologica. Apresentava-se
    como consequências negativas de tais características, o(a) :

    \begin{itemize}
      \item{ Estouro de prazos previamente estipulado }
      \item{ Aumento de custo muito além do planejado }
      \item{ Agravamento da inconsistência entre o que era exigido e o que era provido }
      \item{ Expansão da ingerenciabilidade de projetos em pouco tempo de manutenção }
      \item{ Crescimento da inqualidade do software apresentando falhas, bugs e  }
    \end{itemize}

    % IV. Mitigação da crise e progresso até os dias atuais.
    Apesar da crise em si nunca ter sido absolutamente resolvida de fato, esforços oriundos da experiência
    de vários profissionais e pesquisadores competentes resultaram na consolidação de obras técnicas com orientações
    fundamentais proporcionantes da solução mitigatória sobre tal realidade. Obras como: 

    \begin{itemize}
      \item{ "\textit{Software Engineering: A Report on a Conference Sponsored by the NATO Science Committee} }
      \item{ "\textit{The Mytical Man-Month: Essays on Software Engineering - Frederick P. Brooks Js. }" }
      \item{ "\textit{Refactoring: Improving the Design of Existing Code - Martin Fowler}" }
      \item{ "\textit{Structured Programming - Dijkstra}" }
      \item{ "\textit{Design Patterns: Elements of Reusable Object-Oriented Software - Gamma, Helm, Johnson, Vlissides}" }
      \item{ "\textit{Clean Code - Robert C. Martin}" }
    \end{itemize}

    brilham dentre os esforços escritos em prol de um mundo menos flagelado pelos defeitos
    do software.

    % IV. Crise atual do software. - A dependência
    Dentre os problemas enfrentados pela ciência da computação tem-se o da dependência.
    O software, como solução, surge, da mesma maneira que a pesquisa científica, sobre
    "o ombro de gigantes" pois sua construção se faz de modo a reaproveitar códigos
    (soluções) antes desenvolvidas (conhecidas, no âmbito técnico, como módulos,
    bibliotecas e frameworks).
    
    % V. Importância de reforçar os fundamentos edificados a sangue na primeira grande crise
    % para evitar seu agravamento.
    

    
% Nesta seção devem ser introduzidos o ambiente (\textit{i.e.}Contexto) em que seu trabalho está inserido, além do problema que será abordado no seu trabalho.

\subsection{\textbf{Contextualização e Problema}}
    \label{subsec:contextualizacao-problema}

    O desenvolvimento de software atualmente é impelido, na maioria das organizações,
    por times ágeis auto gerenciados auxiliados pela maravilha desta terceira década:
    A inteligência artificial generativa. Em tal contexto de auxilio constante algoritmos
    não deterministicos pode-se acompanhar o desenrolar de um cenário novo onde a
    qualidade dos sistemas desenvolvidos são, como nunca visto antes, fortemente
    influenciadas tais ferramentas levando a um desfecho ainda incerto.

    Tem-se como certo que problemas de qualidade implementacional originadas pela terceirização
    indiscriminada de boa parte da responsabilidade de pensar e agir em desenvolvendo um software
    à \ac{ia} podem levar a uma crise relativamente semelhante a que acontecera
    durante a crise do software das décadas de 1960 e 1980.

    Nunca foi tão importante como agora a necessidade pela plurificação de trabalhos que
    visem reforçar a exemplificação do emprego de conhecimentos filosófico-técnicos que
    interfiram positivamente na entrega de valor do software hodiernamente.

    Tendo em vista a realidade corrente, determina-se como problema cerne desta pesquisa: Quais
    são os efeitos, sejam eles adversos ou benéficos, de se implementar software
    cuja suas regras de negócio sejam totalmente independentes de bibliotecas e frameworks
    providas por empresas ou grupos colaborativos de programadores ?

    


    % Objetivos (geral e específicos)
    
% Esta seção contém os objetivos de sua pesquisa, contemplando o objetivo principal e as atividades para que este objetivo seja atingido.

\subsection{\textbf{Objetivos}}
    \label{sec:objetivos}
    
    \textit{Esta seção contém os objetivos de sua pesquisa, contemplando o objetivo principal
    e as atividades para que este objetivo seja atingido.}

    Este trabalho tem como objetivo principal pertencer a um conjunto de trabalhos, 
    escritos por outros autores, voltados à exemplificação do emprego total ou quase
    total de padrões arquiteturais sob as restrições de uma tecnologia e sua(s) sub ferramenta(s)
    comercialmente bem sucedidas e largamente utilizadas.
      Sua natureza é aplicada em um projeto real de código fonte aberto e seus objetivos são
    alcaçados por meio da exploração de um cenário cujo o desfecho se orienta pelos axiomas,
    postulados e princípios construidos ao longo de décadas de experiência de autores/programadores
    mundo afora expostos em obras literárias recentes.
      

    % Delimitação do estudo
    %
% Nesta seção deve ser apresentado o problema abordado no estudo de forma a especificá-lo dentro do contexto em que está inserido.

\subsection{\textbf{Delimitação do Estudo}}
    \label{sec:delimitacao-estudo}
    
    Nesta seção deve ser apresentado o problema abordado no estudo apresentando exatamente o que será considerado para o mesmo e deixando claro o que não será levado em conta. Considerando um trabalho que fará um estudo comparativo entre algoritmos a partir do desempenho em bases de dados.
    
    Neste trabalho serão observados apenas X técnicas, são elas, Técnica1, Técnica2, ..., TécnicaN, pois são técnicas utilizadas para classificação de dados em bancos de dados não relacionais. Dentro delas, Y parâmetros são abordados e Z não são pelos motivos AC/DC.
    
    % de forma a especificá-lo dentro do contexto em que está inserido.

    % Justificativa
    
% Esta seção trata os motivos pelos quais seu trabalho é relevante

\subsection{\textbf{Justificativa}}
    \label{sec:justificativa}
    
    Este trabalho justifica-se como sendo, no âmbito acadêmico, um esforço experimental
    em prol de obter conclusões restritas sobre uma tecnlogia acerca de um tópico de
    grande interesse na engenharia de software: dependências.

    Quanto ao âmbito pessoal/profissional, este trabalho é uma forma de aprimorar os conjuntos
    de conhecimentos técnico-arquiteturais do docente, permitindo-lhe um acesso mais seguro
    ao mercado de trabalho de TI.

    Trata-se do emprego do conhecimento pré-existente sobre técnicas implementacionais
    em um sistema real de modo a determinar os limites da implementação de aplicações
    com baixo nível de dependência entre a solução e as ferramentas da solução.


    %\begin{enumerate}
    %    \item O que a comunidade acadêmica irá ganhar com seu trabalho?;
    %    \item Qual a razão do seu trabalho ser desenvolvido?;
    %\end{enumerate}
    

    % Apresentação do trabalho
    
% Esta seção trata a construção do documento, especificamente a disposição dos capítulos e uma introdução do que está sendo abordado nestes.

\subsection{\textbf{Apresentação do Trabalho}}
    \label{sec:apresentacao-trabalho}
    
    Esta seção trata a construção do documento, especificamente a disposição dos capítulos e uma introdução do que está sendo abordado nestes.


% Nesta seção devem ser introduzidos o ambiente (\textit{i.e.}Contexto) em que seu trabalho está inserido, além do problema que será abordado no seu trabalho.
\newpage
