
%%%%%%%%%%%%%%%%%%%%%%%%%%%%%%%%%%%%%
%%      Fundamentação Teórica      %%
%%%%%%%%%%%%%%%%%%%%%%%%%%%%%%%%%%%%%

% Deve haver um tópico para cada assunto abordado na sua pesquisa.

\section{\textbf{FUNDAMENTAÇÃO TEÓRICA}}
    \label{cha:fundamentacao-teorica}
    % remove nome do capítulo do cabeçalho

   %A fundamentação teórica é a base de qualquer pesquisa, oferecendo o embasamento
   %conceitual necessário para entender e explorar um determinado tema. Ela consiste
   %na revisão e síntese crítica de teorias, estudos anteriores e informações relevantes
   %que sustentam a investigação em questão. Essa seção é crucial para mostrar a
   %importância e a originalidade do estudo, fornecendo um conjunto de conceitos e
   %ideias que ajudam na análise dos resultados. Modelos teóricos, conceitos-chave,
   %abordagens metodológicas e estudos anteriores são abordados para fornecer um suporte
   %consistente à pesquisa. 

    %INTRODUÇÃO À SEÇÃO

    Esta seção apresenta partes significativas das obras dos autores que contribuiram
    com o a engenharia de software compilando as mais contrapôem-se como solução a
    problemas relacionados aos males da forte dependência.

    % orientação a objetos - pilares

    % arquitetura limpa - frameworks são detalhes
    Quando fala-se de arquitetura, no contexto de desenvolvimento de software, fica claro que
    o objeto de estudo manifesta-se como um conjunto de códigos.

    % padrões: Adaptador, Strategy, soliD injeção de dependência e outros

  




%   PRINCIPAIS AUTORES
%     Bass, Clements, Kazman ( referência teórica fundamental da arquitetura de software )
%       Software Architecture in Practice
%
%     Robert Cecil Martin (Uncle Bob)
%       Clean Architecture, Implementing clean architecture, Clean architecture with .NET
%
%     Willian Vance
%       Clean Architecture
%     
%     Martin Fowler
%       ?
%
%     Erick Evans ( design de domínio )
%       Domain Driven Design
%     Vaughn Vernon
%       Implementing DDD
%
%     Buschmann et all. ( padrões arquiteturais )
%       Pattern-Oriented Software Architecture

%   ESCOLAS DE PENSAMENTO
%   ÁREAS DE APLICAÇÃO



    % ÚLTIMO TÓPICO
    \subsection{\textbf{Trabalhos Relacionados}}
    \label{sec:trabalhos-relacionados}
\newpage
