
%%%%%%%%%%%%%%%%%%%%%%%%%%%%%%%%%%%%%
%%      Fundamentação Teórica      %%
%%%%%%%%%%%%%%%%%%%%%%%%%%%%%%%%%%%%%

% Deve haver um tópico para cada assunto abordado na sua pesquisa.

\section{\textbf{FUNDAMENTAÇÃO TEÓRICA}}
    \label{cha:fundamentacao-teorica}
    % remove nome do capítulo do cabeçalho

   % A fundamentação teórica é a base de qualquer pesquisa, oferecendo o embasamento
   % conceitual necessário para entender e explorar um determinado tema. Ela consiste
   % na REVISÃO E SÍNTESE CRÍTICA de teorias, estudos anteriores e informações relevantes
   % que sustentam a investigação em questão. Essa seção é crucial para mostrar a
   % importância e a originalidade do estudo, fornecendo um conjunto de conceitos e
   % ideias que ajudam na análise dos resultados. Modelos teóricos, conceitos-chave,
   % abordagens metodológicas e estudos anteriores são abordados para fornecer um suporte
   % consistente à pesquisa. 

    %INTRODUÇÃO À SEÇÃO

    Esta seção apresenta uma breve revisão das obras mais importantes de autores
    que contribuiram com os esforços pro desacoplamento e contra efeitos adversos
    das dependências em desenvolvimento de software.

    % orientação a objetos - pilares
    Antes de avançar sobre conceitos arquiteturais e de padrões de projetos, deve-se, primeiro,
    observar aos fundamentos intrínsecos do paradigma sobre o qual dispõe-se a trabalhar sobre.
    
    O conceito de paradigma é introduzido por \citeonline{floyd2007paradigms}
    como sendo: "um padrão, um exemplo com o qual as coisas são feitas". O mesmo autor
    deixara claro, ao discutir acerca dos paradigmas de sua época, que o conceito, no
    âmbito do desenvolvimento de software, consiste na forma como programas são feitos.

    Os paradigmas surgiram concomitantemente com o desenvolvimento de liguagens de
    programação, em especial nas de alto nível que abstraiam a implementação binária
    direta de instruções, possibilitando uma implementação mais humana e menos complexa
    \cite[p.~8-]{Sammet1969languages}.

    Inicialmente há o surgimento, com o desenvolvimento da arquitetura de Von Neumann,
    do primeiro paradigma, o Imperativo que trazia os conceitos fundamentais de estado
    e ação modificante do estado. Sua influência é, até hoje, enorme e serviu de núcleo
    para paradigmas posteriores \cite[p.~1]{jungthon2009paradigmas}.

    Com o aumento de complexidade dos sistemas, surge o paradigma estruturado que definia
    a sequência, iteração e decisão como sendo as partes fundamentais de qualquer programa
    implementável \cite{Dijkstra1972structured}.

    Por fim, a orientação a objetos, inicialmente implantadas nas linguagens Simula(1962) e
    Smalltalk(1972), traz os conceitos de objeto como uma abstração de qualquer elemento
    da vida real que interage separadamente com outros por meio de "mensagens"
    \cite[p.~52-55]{rentsch1982object}.

    O paradigma orientado a objetos originou, de acordo com a síntese de outros autores presente em
    \citeonline{kasture2019pillars}, quatros pilares fundamentais sobre a implementação definidos como:

    \begin{itemize}
      \item{ Abstração \- capacidade de representar um subconjunto de atributos e comportamentos de uma entidade real }
      \item{ Encapsulamento \- controle de acesso externo a atributos e comportamentos privados de um objeto }
      \item{ Herança \- capacidade de extensão por superconjunto de atributos e comportamentos }
      \item{ Polimorfismo \- capacidade de mutação de comportamentos por sobrescrita }
    \end{itemize}

    O paradigma orientado a objetos possibilitou o desenvolvimento de padrões
    de projeto definidos por procedimentos reprodutíveis e de características
    relativamente parecidas que extendem os quatro pilares fundamentais do
    paradigma em prol de resolver problemas oriundos do desenvolvimento orientado
    a objetos \cite[p.~19-20]{gamma1993design}.

    O padrão possui quatro elementos essenciais: Um nome padrão, um problema
    que ele propõe-se a resolver, a especificação da solução que resolve o problema
    e as consequencias de adotá-lo \cite[p.~19 et al.]{gamma1993design}.

    Dentre os padrões orientados a objetos, os que mais se destacam por suas características
    desacoplantes são: \textit{Adapter}, \textit{Facade}, 

    CONTINUA

    Arquiteturalmente houve o desenvolvimento de conceitos promovedores do desacoplamento
    através da segregação de classes de domínio das demais classes importantes para
    o funcionamento do \textit{software} como um todo. \citeonline[52]{evans2004domain}
    propõe um modelo de quatro camadas genericas: Interface de usuário, aplicação como
    coordenador de ações sen conhecimento de regras de negócio, o domínio como o mantenedor
    e guardião das regras de negócio e a infraestrutura (camada inferior que sustenta todas
    as demais superiores.)



%   PRINCIPAIS AUTORES
%     Bass, Clements, Kazman ( referência teórica fundamental da arquitetura de software )
%       Software Architecture in Practice
%
%     Robert Cecil Martin (Uncle Bob)
%       Clean Architecture, Implementing clean architecture, Clean architecture with .NET
%
%     Willian Vance
%       Clean Architecture
%     
%     Martin Fowler
%       ?
%
%     Erick Evans ( design de domínio )
%       Domain Driven Design
%     Vaughn Vernon
%       Implementing DDD
%
%     Buschmann et all. ( padrões arquiteturais )
%       Pattern-Oriented Software Architecture

%   ESCOLAS DE PENSAMENTO
%   ÁREAS DE APLICAÇÃO



    % ÚLTIMO TÓPICO
    \input{textuais/capitulo_2/sub-topicos/TrabalhosRelacionados}
\newpage
