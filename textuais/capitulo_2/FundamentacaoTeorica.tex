
%%%%%%%%%%%%%%%%%%%%%%%%%%%%%%%%%%%%%
%%      Fundamentação Teórica      %%
%%%%%%%%%%%%%%%%%%%%%%%%%%%%%%%%%%%%%

% Deve haver um tópico para cada assunto abordado na sua pesquisa.

\section{\textbf{FUNDAMENTAÇÃO TEÓRICA}}
    \label{cha:fundamentacao-teorica}
    % remove nome do capítulo do cabeçalho

    A fundamentação teórica é a base de qualquer pesquisa, oferecendo o embasamento conceitual necessário para entender e explorar um determinado tema. Ela consiste na revisão e síntese crítica de teorias, estudos anteriores e informações relevantes que sustentam a investigação em questão. Essa seção é crucial para mostrar a importância e a originalidade do estudo, fornecendo um conjunto de conceitos e ideias que ajudam na análise dos resultados. Modelos teóricos, conceitos-chave, abordagens metodológicas e estudos anteriores são abordados para fornecer um suporte consistente à pesquisa.

    % ÚLTIMO TÓPICO
    \subsection{\textbf{Trabalhos Relacionados}}
    \label{sec:trabalhos-relacionados}
\newpage
