
%%%%%%%%%%%%%%%%%%%%%%%%%%%%%%%%%%%%%
%%      Fundamentação Teórica      %%
%%%%%%%%%%%%%%%%%%%%%%%%%%%%%%%%%%%%%

% Deve haver um tópico para cada assunto abordado na sua pesquisa.

\section{\textbf{FUNDAMENTAÇÃO TEÓRICA}}
    \label{cha:fundamentacao-teorica}
    % remove nome do capítulo do cabeçalho

   A fundamentação teórica é a base de qualquer pesquisa, oferecendo o embasamento
   conceitual necessário para entender e explorar um determinado tema. Ela consiste
   na revisão e síntese crítica de teorias, estudos anteriores e informações relevantes
   que sustentam a investigação em questão. Essa seção é crucial para mostrar a
   importância e a originalidade do estudo, fornecendo um conjunto de conceitos e
   ideias que ajudam na análise dos resultados. Modelos teóricos, conceitos-chave,
   abordagens metodológicas e estudos anteriores são abordados para fornecer um suporte
   consistente à pesquisa. 

%    NÚMERO DE ITERAÇÕES REFATORATIVAS NESTA SEÇÃO = 2
%    PALAVRAS CHAVES = [DÉBITO TÉCNICO, DOMÍNIOS, NEGÓCIO, TÉCNICO]

    %INTRODUÇÃO À SEÇÃO

    Esta seção expõe o atual estado da arte sobre conhecimentos nucleares à arquitetura limpa,
    também revela como esta forma de conceber sistemas surgiu no início do século XXI em
    resposta a problemas oriundos de débitos técnicos e problemas de modelagem de domínio.

    Quando fala-se de arquitetura, no contexto de desenvolvimento de software, fica claro que
    o objeto de estudo manifesta-se na forma de varias estruturas escritas e metodicamente
    organizadas.
  
    Em nível superior diz-se que os diretórios (pastas) e arquivos fontes (códigos) neles
    residentes perfazem o conjunto de elementos primordiais de um software. São eles que
    são o objeto de trabalho de times de desenvolvimento. Aqui aplicam-se padrões de organização
    que separam responsabilidades de vários níveis em um sistema %\cite{}.

    Todo código, peça importantíssima de qualquer projeto, é um arquivo e, como tal, apenas difere
    dos demais por, de acordo com \citeonline{IEEEGlossary}
    ser uma definicao estrutura de dados e instruções expressas em uma linguagem de programação com o
    intuito de expressar um programa de computador. De forma mais precisa, trata-se uma
    definição de dados e instruções capazes de serem processados por montadores, compiladores ou interpretadores \cite{IEEEGlossary}[p. 68].

    % ESCREVER INTRODUÇÃO ANTES DE CONTINUAR

    Os conceitos resgatados são de grande importância devido ao fato de que grandes sistemas sempre
    possuirem seus códigos fonte espalhados por vários diretórios e, até os dias atuais, tem-se como
    nebulosa a definição de um componente de um software, também chamado: modulo ou unidade.

    aedifica-se sobre o alicerce padronizado de fundamentos aos quais chamamos paradigma de programação.
    Todo código segue estritamente regras gramaticais e lógicas que foram concebidas de modo a resolver
    deficiências recorrentes em projeto e construção de software.

    Uma das obras mais populares entre programadores, escrita por Robert C. Martin, Arquitetura Limpa (2008),
    já em seu prefácio, define como objetos de discussão arquiteturais os: componentes, classes e módulos
    e aponta, de forma categória, a capacidade de que tais estruturas têm de ter sua complexidade
    multiplicada de várias formas em vários contextos \cite{}.
        
    %HISTÓRICO

    Antes de apresentar a realidade presente é necessário, primeiramente, apresentar a
    realidade como ela fora e como ela levou a criação do conjunto de conhecimentos abrigados
    sob a égide do nome "arquitetura limpa".
      Assim como várias outras visões arquiteturais desenvolveram-se iterativamente ao longo do tempo,
    a arquitetura limpa teve, em sua história, várias etapas evolucionais que, inclusive, nomearam-se de
    formas diferentes.
    
    ...



%   PRINCIPAIS AUTORES
%     Bass, Clements, Kazman ( referência teórica fundamental da arquitetura de software )
%       Software Architecture in Practice
%
%     Robert Cecil Martin (Uncle Bob)
%       Clean Architecture, Implementing clean architecture, Clean architecture with .NET
%
%     Willian Vance
%       Clean Architecture
%
%     Erick Evans ( design de domínio )
%       Domain Driven Design
%     Vaughn Vernon
%       Implementing DDD
%
%     Buschmann et all. ( padrões arquiteturais )
%       Pattern-Oriented Software Architecture

%   ESCOLAS DE PENSAMENTO
%   ÁREAS DE APLICAÇÃO



    % ÚLTIMO TÓPICO
    \input{textuais/capitulo_2/sub-topicos/TrabalhosRelacionados}
\newpage
