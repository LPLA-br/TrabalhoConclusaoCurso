%%%%%%%%%%%%%%%%%%%%%%%%%%%%%%%%%%%%%%%%%%%
%%      Desenvolvimento da Pesquisa      %%
%%%%%%%%%%%%%%%%%%%%%%%%%%%%%%%%%%%%%%%%%%%

%% Neste capítulo expõe-se os experimentos da
%% sua pesquisa a proposta de solução da sua
%% pesquisa e em caso do TCC II como ela foi implementada

\section{\textbf{DESENVOLVIMENTO DA PESQUISA}}
    \label{sec:desenvolvimento-da-pesquisa}
    % remove nome do capítulo do cabeçalho
    %\input{textuais/capitulo_4/sub-topicos/PropostaDeSolucao}

    %Introdução ao desenvolvimento (do tipo como foi implementado)

    % para quem foi desenvolvido este sistema ?
    O sistema objeto surge como uma solução para gerência de processos em
    ateliês de costura. Em síntese, o domínio de negócios estudado é: local; de
    impacto limitado; informal e executado por uma pessoa.

    Em virtude disto, partiu-se do pressuposto de que todo empreendimento de
    costura informal possui problemas comuns de ingerência de suas informações.
    Sendo assim, a partir deste pressuposto estabeleceu-se um canal de
    comunicação coloquial com profissional com anos de experiência de atuação
    autônoma próxima ao programador.

    % qual é o problema ?
    Ao principiar-se no levantamento de requisitos, obteve-se um conjunto de
    caracteristicas que, em confirmação do que fora supra exposto, um
    \textit{software} voltado para o domínio de costura e conserto de roupas
    deveria ter.

    Como consequência da comunicação inicial revelou-se situações problemáticas
    onde: os prazos de serviços são esquecidos senão o próprio serviço e suas
    características; o levantamento de custos com insumos usados em confecções
    e concertos não monitorado.

    % com o que resolver o problema ?
    Por fim, ao determinar problemas, segue-se para a escolha de quais tecnologias
    devem ser empregadas em função do que o contexto apresentou como desafio. A estes
    atribui-se o título de requisitos não funcionais.

    Contudo, em função do orçamento extremamente limitado não houve
    possibilidade de construção de um sistema seguro o suficiente para operar
    redes abertas WAN. Logo determinou-se que a aplicação objeto seria de
    natureza local à rede dos estabelecimentos, abrindo espaço para
    flexibilizações e simplificações não adequadas ou possíveis a sistemas
    presentes na \ac{www}.

    % ...
    \begin{figure}[h]
      \centering
      \caption{ Diagrama da Infraestrutura Local de Rede. }
      \includegraphics[scale=0.5]{lib/nodes-uml.png}
      \source{ O Autor {\imprimirdata}}
      \label{fig:figura}
    \end{figure}

    %\input{textuais/capitulo_4/sub-topicos/Experimentos}
    \newpage
