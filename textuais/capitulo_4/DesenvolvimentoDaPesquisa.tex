%%%%%%%%%%%%%%%%%%%%%%%%%%%%%%%%%%%%%%%%%%%
%%      Desenvolvimento da Pesquisa      %%
%%%%%%%%%%%%%%%%%%%%%%%%%%%%%%%%%%%%%%%%%%%

%% Neste capítulo expõe-se os experimentos da
%% sua pesquisa a proposta de solução da sua
%% pesquisa e em caso do TCC II como ela foi implementada

\section{\textbf{DESENVOLVIMENTO}}
    \label{sec:desenvolvimento-da-pesquisa}
    % remove nome do capítulo do cabeçalho
    %\input{textuais/capitulo_4/sub-topicos/PropostaDeSolucao}

    % para quem foi desenvolvido este sistema ?
    O domínio de negócios do objeto surge como uma solução para gerência
    de processos em ateliês de costura. Em síntese, ele possui características:
    ser local e causar impacto limitado em ambiente informal.

    Em virtude disto, partiu-se do pressuposto de que todo empreendimento de
    costura informal possuiria problemas comuns de ingerência de suas
    informações. Sendo assim, a partir deste pensamento estabeleceu-se um canal
    de comunicação coloquial entre profissional com anos de experiência de
    atuação autônoma na área e programador.

    % qual é o problema ?
    A principiar-se o levantamento de requisitos, se obteve um conjunto de
    caracteristicas que definiu, em confirmação do que fora presumido, um
    \textit{software} que deveria ser voltado para o domínio de costura e
    conserto de roupas.

    Como consequência da comunicação inicial, revelou-se situações problemáticas
    onde os prazos de serviços são esquecidos, senão o próprio serviço e suas
    características, bem como o levantamento de custos com insumos usados em
    confecções e concertos não é monitorado.

    %aspectos não funcionais
    Em seguida, partiu-se para escolha das melhores ferramentas para o contexto
    desafiador que delineou-se a problemática. Comessou, aqui, a etapa que avalia
    requisitos/aspectos não funcionais da aplicação.

    Neste sentido, \citeonline[p.85]{somervilleIam2011} define os requisitos não funcionais
    como sendo aqueles que não atribuem funcionalidades à aplicação, que
    são fundamentais para existência dela. Eles são: o ambiente físico,
    as questões de segurança da informação, a capacidade física da máquina
    servidora dentre outros que as circunstâncias possam trazer.

    Assim, o ambiente do sistema, de modo a garantir acessibilidade por
    qualquer rede com acesso a internet, inicialmente fora desenvolvido para
    ser acessível por redes de internet de amplo alcançe, conhecidas como \ac{wan}.
    Entretanto, viu-se que essa caracteristica de disponibilidade ampla não
    seria viável devido ao alto custo de infraestrutura de computadores em
    núvem e possíveis problemas legais decorrentes de vazamentos de dados
    \cite{protecaoDados2018}.

    Além disso, considerando o foco experimental assumido, estabeleceu-se o
    ambiente de funcionamento estrito à redes locais \ac{lan} cuja disponibilidade
    fica espacialmente restrita ao estabelecimento. Abordagem esta, semelhante
    a encontrada em aplicações anteriores à difusão da acessibilidade da
    internet.

    % ...
    \begin{figure}[h]
      \centering
      \caption{ Diagrama da Infraestrutura Local de Rede Simples. }
      \includegraphics[scale=0.5]{lib/nodes-uml.png}
      \source{ O Autor {\imprimirdata}}
      \label{fig:figura}
    \end{figure}

    %php como linguagem.
    Conseguinte, prosseguiu-se no emprego de uma linguagem de programação
    interpretada cuja caracteristica mais importante fosse a capacidade de
    integrar-se bem com o servidor. Logo figurou-se opções como: Javascript;
    PHP; Python; Ruby; Rust; Go dentre outras. Entretanto, devido ao já largo
    emprego de Javascript, assumiu-se uma alternativa mais tradicional na
    linguagem de programação PHP.
    
    O PHP, que nascera como um conjunto de binários escritos na linguagem C,
    tornou-se, na década de 2000 e 2010, uma das mais utilizadas ferramentas do
    desenvolvimento \textit{backend} \cite{phpHistory}. Apesar de seu constante
    declínio atual em face de opções mais modernas, sua influência ainda não é
    negligenciável pois ainda se encontra presente no cerne de inúmeras aplicações
    legadas e \textit{software} \ac{web} \cite{theRegisterPHPDecline}.

    %modelo de classes
    Neste sentido, assumimos uma linguagem de programação para a implementação
    do projeto, adiantando-se para uma modelagem do esquema de classes do
    domínio de negócios, onde o domínio minunciosamente estudado trás as
    seguintes classes fundamentais para solução do problema de negócio.

    \begin{figure}[h]
      \centering
      \caption{ Diagrama Textual Indentado das Classes Principais. }
      \includegraphics[scale=0.8]{lib/classes-principais.png}
      \source{ O Autor {\imprimirdata}}
      \label{fig:figura}
    \end{figure}

    Complementando o diagrama sobreposto, determina-se que "ServiçoCostureira",
    elemento pivô, possui relação de composição de clientes que por sua vez
    agregam classes representantes de seus dados. Por outro lado há classe
    de Coleção de peças, também parte composta de serviço que, por conseguinte,
    possui de sua parte uma coleção de insumos e seus respectivos dados.

    Em questão de organização de diretórios e sub diretórios, convencionou-se a
    criação da seguinte hierarquia:

    \begin{itemize}
      \item{ Collection - Classes que gerenciam plúrimos objetos. }
      \item{ Entity - Classes que representam Entidades do domínio de negócios. }
      \item{ Service - Classes que implementam casos de uso do sistema instanciadas em main. }
      \item{ Repository - Classes que manipulam banco de dados através de interface padronizada. }
      \item{ Comunication - Classes que usam interface udp/ip para comunicação em redes "\textit{project's specific}". }
      \item{ Enum - Enumerações que representam estados evitando strings mágicas. }
      \item{ ValueObject - Objetos de valor que não são omnirelevantes ao domínio }
    \end{itemize}

    Tal organização orientou-se através de \citeonline{
      goldberg1984smalltalk, evans2004domain, uncleBob2017 }

    No tocante às classes coleções, a obra \citeonline{goldberg1984smalltalk}
    foi de extrema importância na orientação cautelosa de se evitar que objetos
    como peças e insumos possuissem excesso de responsabilidades em manipulando
    matrizes de subobjetos agregados.

    \begin{figure}[h]
      \centering
      \caption{ Classes Coleção. }
      \includegraphics[scale=0.8]{lib/colecoes.png}
      \source{ O Autor {\imprimirdata}}
      \label{fig:figura}
    \end{figure}

    %módulos do PHP e adaptadores
    Com relação às dependências, que são foco desse experimento, o PHP possui
    extensões padrões ativáveis na sua configuração, onde cada módulo extende sua
    capacidade de modo a permitir que a aplicação possa performar ações
    críticas às funcionalidades instituidas nos requisitos funcionais.

    %pgsql e sockets
    Duas extensões cuja finalidades garatem as capacidades
    de execução de transações \ac{sql} e atendimento do ciclo requisição/resposta
    foram ativadas, respectivamente sendo elas: pgsql e sockets.

    %respectivos adaptadores
    No âmbito dos dados, a ação tomada objetivou a independência da ferramenta
    que manipula o banco de dados através da criação de uma interface, também
    conhecida como contrato abstrato, onde definiria que classes forasteiras
    estaria em conformidade com tal contrato que, a princípio, possuiria quatro
    métodos fundamentais escritos, sendo eles: ler, escrever, redefinir e
    encerrar.

    Em seguida, escreveu-se uma classe do tipo adaptador que implementa o
    sobrecitado contrato, onde esse código seria capaz de acessar diretamente a
    \ac{api} da extensão pgsl. Ademais, esta abordagem também visou alcançar
    a simplicidade através de um adaptador que pudesse ser universalmente
    utilizado por classes do tipo repositório possuidoras de código específico
    de consultas para instânciação de classes de domínio \ac{sql}.

    A posteriori, viu-se as ações assumidas, no contexto dos dados, criaram
    um cenário com características próximas mas não idênticas às expostas por
    \citeonline[p.106]{evans2004domain} na seção que trata acerca de repositórios,
    conforme o diagrama:

    \begin{figure}[h]
      \centering
      \caption{ Adaptador de Banco de Dados é, Por Contrato, Substituível. }
      \includegraphics[scale=0.5]{lib/adaptador-dados.png}
      \source{ O Autor {\imprimirdata}}
      \label{fig:figura}
    \end{figure}

    Quanto ao âmbito da comunicação, não utilizou-se protocolo de aplicação
    em redes conhecido como \ac{http}, como consequência, elaborou-se um protocolo
    próprio construido sobre o \ac{udp}. No que se refere ao protocolo ora apresentado,
    trata-se de comandos na forma de substantivo, verbo e dados da ação, conforme
    exemplificado no protocolo desenvolvido:

    \begin{figure}[h]
      \centering
      \caption{ Exemplos de Comandos do Protocolo Atelie. }
      \includegraphics[scale=0.5]{lib/protocoloatelie.png}
      \source{ O Autor {\imprimirdata}}
      \label{fig:figura}
    \end{figure}
    
    Partindo desse modelo, vê-se que para cada evento de requisição do cliente,
    há ação por parte de um servidor, do direcionamento para o tratamento do
    comando, bem como, para o roteamento em casos de uso e resposta.

    Quanto ao módulo dependido para esta funcionalidade, a questão da
    independência foi garantida através da implementação de uma classe chamada
    servidor. Sendo ela implementante de uma interface de mesmo nome que define
    métodos padrões reconhecidos pelo ponto de execução principal "main.php", conforme
    a diagramação que se segue:

    \begin{figure}[h]
      \centering
      \caption{ Relação Servidor, IServidor e Main. }
      \includegraphics[scale=0.5]{lib/comunicacao.png}
      \source{ O Autor {\imprimirdata}}
      \label{fig:figura}
    \end{figure}

    %funcionamento geral/ estado do sistema 
    Por fim, conclui-se a etapa de construção, onde a comunicação e os
    dados são funcionais e relativamente mais independentes entre eles, onde
    podemos extrair resultados e conclusões sobre os mesmos.

    %\input{textuais/capitulo_4/sub-topicos/Experimentos}
    \newpage
