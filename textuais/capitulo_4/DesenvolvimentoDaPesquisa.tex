%%%%%%%%%%%%%%%%%%%%%%%%%%%%%%%%%%%%%%%%%%%
%%      Desenvolvimento da Pesquisa      %%
%%%%%%%%%%%%%%%%%%%%%%%%%%%%%%%%%%%%%%%%%%%

%% Neste capítulo expõe-se os experimentos da
%% sua pesquisa a proposta de solução da sua
%% pesquisa e em caso do TCC II como ela foi implementada

\section{\textbf{DESENVOLVIMENTO DA PESQUISA}}
    \label{sec:desenvolvimento-da-pesquisa}
    % remove nome do capítulo do cabeçalho
    %\input{textuais/capitulo_4/sub-topicos/PropostaDeSolucao}

    % para quem foi desenvolvido este sistema ?
    O sistema objeto surge como uma solução para gerência de processos em
    ateliês de costura. Em síntese, o domínio de negócios estudado é: local; de
    impacto geograficamente limitado a pequena cidade; informal e exercido por
    uma pessoa.

    Em virtude disto, partiu-se do pressuposto de que todo empreendimento de
    costura informal possuiria problemas comuns de ingerência de suas
    informações. Sendo assim, a partir deste pensamento estabeleceu-se um canal
    de comunicação coloquial com profissional com anos de experiência de
    atuação autônoma na área.

    % qual é o problema ?
    Ao comessar o levantamento de requisitos, obteve-se um conjunto de
    caracteristicas que, em confirmação do que fora presumido, um
    \textit{software} voltado para o domínio de costura e conserto de roupas
    deveria ter.

    Como consequência da comunicação inicial, revelou-se situações problemáticas
    onde: os prazos de serviços são esquecidos senão o próprio serviço e suas
    características; o levantamento de custos com insumos usados em confecções
    e concertos não é monitorado.

    % com o que resolver o problema ?
    Em seguida, com problemas determinados, segue-se para a escolha de quais
    tecnologias devem ser empregadas em função do que o contexto apresentou
    como desafio. A estes atribui-se o título de requisitos não funcionais.

    \citeonline[p.244]{rogerPressman2021} define os requisitos não funcionais
    como sendo aqueles que não atribuem funcionalidades à aplicação, mais 
    são fundamentais para existência dela.

    Assim, o desenvolvimento do sistema voltou-se para acessibilidade por
    qualquer rede com acesso a internet. Entretanto, viu-se que essa
    caracteristica não seria viável devido ao: Alto custo de infraestrutura
    de computadores em núvem; e possíveis problemas legais decorrentes de
    vazamentos de dados.

    Além disto, considerando o foco experimental assumido, estabeleceu-se
    o ambiente de funcionamento estrito à redes locais LAN. Abordagem esta,
    semelhante à encontrada em aplicações anteriores à difusão da acessibilidade
    da internet.

    % ...
    \begin{figure}[h]
      \centering
      \caption{ Diagrama da Infraestrutura Local de Rede. }
      \includegraphics[scale=0.5]{lib/nodes-uml.png}
      \source{ O Autor {\imprimirdata}}
      \label{fig:figura}
    \end{figure}

    %php como linguagem.
    Dessa forma, empregar uma linguagem de programação foi mandatório
    na necessidade de se escrever o código da aplicação. Dentre
    as várias opções disponíveis no mercado, convencionou-se o uso de PHP.

    A linguagem, que nascera como um conjunto de binários escritos na linguagem
    C, tornou-se, na década de 2000 e 2010, uma das mais utilizadas ferramentas
    do desenvolvimento web backend. Apesar de seu constante declínio em face de opções
    mais modernas como Javascript, sua proeminência não é negligenciável.

    %uso de módulos do php.
    %


    %\input{textuais/capitulo_4/sub-topicos/Experimentos}
    \newpage
