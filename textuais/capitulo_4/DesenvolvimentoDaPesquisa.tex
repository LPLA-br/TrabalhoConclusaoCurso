%%%%%%%%%%%%%%%%%%%%%%%%%%%%%%%%%%%%%%%%%%%
%%      Desenvolvimento da Pesquisa      %%
%%%%%%%%%%%%%%%%%%%%%%%%%%%%%%%%%%%%%%%%%%%

%% Neste capítulo expõe-se os experimentos da
%% sua pesquisa a proposta de solução da sua
%% pesquisa e em caso do TCC II como ela foi implementada

\section{\textbf{DESENVOLVIMENTO DA PESQUISA}}
    \label{sec:desenvolvimento-da-pesquisa}
    % remove nome do capítulo do cabeçalho
    %\input{textuais/capitulo_4/sub-topicos/PropostaDeSolucao}

    Detalhar prática da execução.

    E é a partir do levantamento de requisitos que inicia-se as etapas de
    desenvolvimento de uma aplicação voltada para micro empreendedoras costureiras.

    Partiu-se do pressuposto de que todo empreendimento de costura informal
    poderiar possuir problemas comuns de ingerência de suas informações. A
    partir deste pressuposto estabeleceu-se um canal de comunicação coloquial
    com profissional com anos de experiência de atuação autônoma próxima
    ao autor.

    A comunicação inicial revela situações problemáticas onde: os prazos são
    esquecidos, o levantamento de custos com insumos usados em confecções e
    concertos é ignorado e armazenamento irregular de medidas corporais por
    vezes dificulta consultas rápidas e confunde quando em consultado
    dados obsoletos.

    % ...
    \begin{figure}[h]
      \centering
      \caption{ Diagrama da Infraestrutura Local de Rede. }
      \includegraphics[scale=0.5]{lib/nodes-uml.png}
      \source{ O Autor {\imprimirdata}}
      \label{fig:figura}
    \end{figure}

    Quanto aos requisitos não funcionais, concluiu-se que sistema atuante em
    redes locais seria, em materia de confiabilidade, custo e legalidade, mais
    adequado para estes ambientes de processos simples.
    
    %\input{textuais/capitulo_4/sub-topicos/Experimentos}
    \newpage
