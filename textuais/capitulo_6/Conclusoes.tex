%%%%%%%%%%%%%%%%%%%%%%%%%%%%
%%       Conclusões       %%
%%%%%%%%%%%%%%%%%%%%%%%%%%%%

\section{\textbf{CONCLUSÃO}}
    \label{sec:conclusao}
    % remove nome do capítulo do cabeçalho
    
    O sistema Atelie constituiu-se como reforço na compreensão
    e de técnicas, filosofias e arquiteturas por parte do
    autor. Apesar da sua natureza experimental, suas nuances
    trouxeram luz às questões levantadas nos objetivos deste trabalho.

    Diante dessa premissa é possível implementar software que independe de partes satelites
    conferintes de funcionalidade. Entretanto, percebeu-se que abordagens que
    visam tal objetivo consomem mais recursos intelectuais e temporais por
    parte do técnico responsável, ao mesmo tempo que abrem espaço para o emprego
    desenfreado de complexidade desnecessária, problema este conhecido como
    \textit{overengeneering} "excessos de engenharia".

    Ademais observou-se também que classes e funções embutidas na
    linguagem adotada surgem como ponto de inflexão, remedia-las
    é um esforço desnecessariamente árduo e com capacidade de implicar em
    gargalos de desempenho, atrasos na entrega de valor e falhas e erros em
    problemas difíceis de detectar-las.

    Por fim, conclui-se que a observância das características do domínio em
    conjunto com consulta crítica de obras técnicas, abre caminho para
    consolidação de um projeto tecnicamente equilibrado e capaz de perpassar
    ao teste do tempo.



    %\input{textuais/capitulo_6/sub-topicos/Discussao}
    %\input{textuais/capitulo_6/sub-topicos/Contribuicoes}
    %\input{textuais/capitulo_6/sub-topicos/Limitacoes}
    %\input{textuais/capitulo_6/sub-topicos/TrabalhosFuturos}
    \newpage
