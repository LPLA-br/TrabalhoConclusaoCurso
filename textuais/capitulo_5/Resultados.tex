
%%%%%%%%%%%%%%%%%%%%%%%%%%%%%%%%%%%%%%%%%%
%%          Resultados Obtidos          %%
%%%%%%%%%%%%%%%%%%%%%%%%%%%%%%%%%%%%%%%%%%


\section{\textbf{RESULTADOS}}
    \label{sec:resultados}
    % remove nome do capítulo do cabeçalho

    A presente seção aponta os resultados práticos da implementação
    do sistema na forma de facilidades e dificuldades.

    Dente estas sub partes, apontamos que as interfaces presentes em
    \textit{Comunication} e \textit{Repository} permitem que o programador crie
    classes que as implementam. Tal abordagem, que vai de encontro com o
    princípio de inversão de dependência presente dos princípios SOLID, que garante que há
    alguma base para substituição de componentes externos ao domínio sem
    alterações profundas nos internos.

    A abordagem mista e entre herança e composição resultou em um modelo
    de classes de granularidade facilemente compreensível e gerenciável
    ao programador, não havendo rigidez excessiva do uso abusivo de herança
    e nem o caus de inúmeras interfaces sendo criadas para cada relação de
    composição.

    %Neste capítulo são expostos os resultados de sua pesquisa.
    %No caso do TCC I, resultados esperados ou parciais,
    %para TCC II, resultados "finais".

\newpage
