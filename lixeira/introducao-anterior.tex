
    % Contextualização e problema
    No século XXI o software é o mais importante produto da industria 4.0 e 5.0 .
    Vários setores da economia conteporânea apoiam sua logistica de produção de produtos e 
    provimentos de serviços por meio de tais ferramentas (expandir).

    Nestes primeiros anos da década de 2020 viu-se, não somente no
    mercado nacional, o pico na demanda por profissionais com
    competências para construir soluções em \ac{ti}, demanda esta justificada pelos
    problemas sanitários do período pandêmico que limitaram o contato humano e favoreceram a expansão
    da infraestrutura digital em campos como a educação, comunicações,
    comércio digital e outros \cite{vieira2024impactos} e \cite{carreira2023}.
    
    Entretanto, o que se seguiu foi um cenário de demissões em massa e uma aparente
    queda de demanda por profissionais, o periódico  \citeonline{carreira2023} aponta
    como principais motivos o pós pandemia, que trouxe redução de demanda justificada
    pelos novos níveis de receitas das companhias de tecnologia e a recessão global
    que dá sinais de crise a bastante tempo.

    Atualmente, 50\% profissionais apontam salários menores e
    jornadas mais exaustivas enquanto 45\% empresas, demandantes de habilidades mais
    concretamente estabelecidas, enfretam dificuldades para contratar profissionais
    realmente qualificados \cite{cnn2024}.

    A dificuldade e necessidade em conseguir profissionais qualificados apontam para
    um cenario onde a entrega de valor com qualidade é uma prioridade maior em tempos de
    estabilização do mercado e garanti-la através de profissionais mais capazes
    em evitar os efeitos do débito técnico, comumente associados à falta de experiência,
    é nuclear \cite[p.~131 et all.]{beltrao2020}. Além disto o advento da \ac{ia} possibilitou
    a automação parcial ou completa de processos simples que antes eram exercidos por
    profissionais menos experientes como bem demonstrado por \citeonline{baptista2023ia}: Ele
    apontara capacidades de, com prompts simples, construir códigos de calculadora
    e sua interface funcionais, estilização básica consistente e até mesmo um quadro de desenhos
    reinicialiável.

    Quanto a métricas de geração de valor no processo de transformação digital,
    \citeonline[p. 9]{metricasResende2024}, ao avaliar grandes empresas Brasileiras
    do ramo atuantes no Rio de Janerio, inclui, dentre um conjunto de desafios, como
    sendo grandes detratores da geração de valor: Superar desalinhamento entre Tecnologia e negócio
    e deter todas competências de \ac{ti} necessárias.

    Tanto o desalinhamento entre negócio e \ac{ti} quanto o acesso limitado de profissionais capacitados
    evidenciam uma carência de conhecimentos arquiteturais e de implementação voltada a
    representar o domínio de negócios.

    \citeonline[p.~131 et all.]{beltrao2020} introduz o débito técnico como
    sendo o foco em benefícios de curto prazo acaba criando problemas na manutenibilidade
    de um sistema implicando em uma realidade de custos maiores e, em casos extremos,
    de re-implementações totais de novos sistemas. Esta ocorrência, não incomum em
    projetos cujos os prazos são curtíssimos, têm como contributor principal
    oriundo da pressa, o emprego impudente de soluções não alinhadas com princípios, 
    fundamentos e bons hábitos na implementação técnica de softwares.
