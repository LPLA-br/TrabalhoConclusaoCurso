
  % ESCREVER INTRODUÇÃO ANTES DE CONTINUAR

  Os conceitos resgatados são de grande importância devido ao fato de que grandes sistemas sempre
  possuirem seus códigos fonte espalhados por vários diretórios e, até os dias atuais, tem-se como
  nebulosa a definição de um componente de um software, também chamado: modulo ou unidade.

  aedifica-se sobre o alicerce padronizado de fundamentos aos quais chamamos paradigma de programação.
  Todo código segue estritamente regras gramaticais e lógicas que foram concebidas de modo a resolver
  deficiências recorrentes em projeto e construção de software.

  Uma das obras mais populares entre programadores, escrita por Robert C. Martin, Arquitetura Limpa (2008),
  já em seu prefácio, define como objetos de discussão arquiteturais os: componentes, classes e módulos
  e aponta, de forma categória, a capacidade de que tais estruturas têm de ter sua complexidade
  multiplicada de várias formas em vários contextos \cite{}.
      
  %HISTÓRICO

  Antes de apresentar a realidade presente é necessário, primeiramente, apresentar a
  realidade como ela fora e como ela levou a criação do conjunto de conhecimentos abrigados
  sob a égide do nome "arquitetura limpa".
    Assim como várias outras visões arquiteturais desenvolveram-se iterativamente ao longo do tempo,
  a arquitetura limpa teve, em sua história, várias etapas evolucionais que, inclusive, nomearam-se de
  formas diferentes.
  
  ...
